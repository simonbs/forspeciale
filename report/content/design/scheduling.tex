\section{Scheduling}\label{sec:desScheduling}
Since we rely on flexible consumption and because it is possible to forecast production, we can schedule the flexible consumption to match the forecasted production. Flexible consumption is consumer energy consumption that is not necessarily needed right now. The goal of scheduling is to move this type of energy consumption below the produced energy, as shown in \Cref{sec:solapproach} and thus increase the profit for the BRP by not needing to purchase energy unless required. The flexible consumption in is purely based on charging of electric vehicles.  

\subsection{Idea}\label{sec:idea}
The idea is to generate a table representing a schedule for charging EVs as a result of scheduling (design of a ChargePlan is shown in \Cref{fig:UMLCP}). \Cref{tab:sched} shows such a table, highlighting which time slots an EV should charge in, represented by a boolean variable. It takes on the value ``\charge{}'' when the EV should charge, and ``\nocharge{}'' when it should not charge. Rows represent charge-offers, and columns represent time slots. In order to account for EVs that are plugged in after this schedule is generated, we iteratively generate a completely new schedule each time a new time slot is reached.

We decide to make the input and output for each type of scheduler to be the same, allowing us to implement different types of schedulers. The input required by these schedulers is a charge-context, i.e.\ a list of the forecasted production, a list of charge-offers and the prices for importing/exporting for each time slot (see \Cref{fig:UMLCC}). This input can be illustrated as a table shown below in \Cref{tab:unsched}. Note that this table is \emph{not} a schedule, but an input to scheduling. Each cell represents the energy that a charge-offer's EV uses if it charges during that time slot. The number of non-zero values in a row represents how many times lots the EV may charge; an EV may not charge after its deadline, represented by \nocharge{}'s. E.g.\ the EV represented by charge-offer $\var{CO}_1$ may charge during the first three time slots and must not charge in the last three, due to $T_4$ being its deadline.

\begin{table}[htpb]
  \centering
  \begin{tabular}{l S S S S S S} \toprule
    \var{CO} & {$T_1$}   & {$T_2$}  & {$T_3$}  & {$T_4$}   & {$T_5$}   & {$T_6$} \\ \midrule
    $\var{CO}_1$    &    10     &    10    &    10    &     0     &     0     &     0   \\
    $\var{CO}_2$    &     8     &     8    &     8    &     8     &     0     &     0   \\
    $\var{CO}_3$    &     7     &     7    &     7    &     7     &     7     &     7   \\
    $\var{CO}_4$    &     4     &     4    &     0    &     0     &     0     &     0   \\
    $\var{CO}_5$    &     5     &     5    &     5    &     5     &     5     &     0   \\
    $\var{CO}_6$    &     6     &     6    &     6    &     6     &     0     &     0   \\ \midrule
       \var{Prod}   &    40     &    50    &    30    &    20     &    40     &    35   \\ \bottomrule
  \end{tabular}
  \caption{An example of a charge-context, the input to a scheduler, based on \Cref{fig:UMLCC}}\label{tab:unsched}
\end{table}

\begin{table}[htpb]
\centering
  \begin{tabular}{l S S S S S S} \toprule
    \var{CO} & {$T_1$} & {$T_2$} & {$T_3$} & {$T_4$} & {$T_5$} & {$T_6$} \\ \midrule
    $\var{CO}_1$ & \charge{}    & \charge{}  & \charge{} & \nocharge{}  & \nocharge{}  & \nocharge{} \\
    $\var{CO}_2$ & \charge{}   & \charge{}  & \nocharge{}  & \charge{} & \nocharge{}  & \nocharge{} \\
    $\var{CO}_3$ & \charge{}    & \charge{}  & \charge{} & \charge{} & \charge{} & \charge{} \\
    $\var{CO}_4$ & \charge{}    & \charge{}  & \nocharge{}  & \nocharge{}  & \nocharge{}  & \nocharge{} \\
    $\var{CO}_5$ & \nocharge{}  & \nocharge{}   & \charge{} & \charge{} & \charge{} & \nocharge{} \\
    $\var{CO}_6$ & \nocharge{}  & \charge{}  & \charge{} & \nocharge{}  & \nocharge{}  & \nocharge{} \\ \bottomrule
	\end{tabular}
	\caption{Example of a charge-plan, the complete charging schedule, based on \Cref{fig:UMLCP}}\label{tab:sched}
\end{table}

Time slots will have a duration of 20 minutes each in our system. But we design it so that this length is variable throughout the entire system, allowing the BRP to fit it to their needs. They may need to change this length due to varying factors, such as scheduling taking too long with one method in lieu of another on their computers or that a lower value is desired so that more detailed schedules are produced. The duration of \SI{20}{\minute} is chosen with regards to the following arguments:
\begin{itemize}
  \item An hour is dividable by 20 and it follows that an entire day is dividable by 20. Thus we can generate schedules for whole days without having any offsets.
  \item The longer a time slot is, the more time exists to generate a schedule
  \item The shorter a time slot is, the more often it is possible to check for newly plugged in EVs, which allows earlier charging of these EVs, possibly saving even more energy
  \item The longer\slash shorter a time slot is, the more complex\slash simple the schedule becomes. Longer time slots will require fewer schedules during a day, but the schedules are also less accurate. Shorter time slots will increase the number of schedules during a day, but will also generate more accurate schedules.
\end{itemize}

20 minutes for each duration is chosen because it seems like a long enough duration to generate schedules, but without being too long to be good at scheduling newly plugged-in EVs. Our scheduling is thus iterative as seen by \Cref{fig:schedspiral}.

\begin{figure}[!htb]
  \centering
  \input{drawings/iterativescheduling}
  \caption{Iterative scheduling}\label{fig:schedspiral}
\end{figure}

An alternative to re-scheduling every time slot would be to reschedule every time an EV is plugged in. This will, however, require enormous computational power if the schedule is large. It would, however, allow charging of EVs exactly when they are plugged in, but is not possible due to the time it takes to reschedule. 

\subsection{Optimization Problem} \label{sec:optimizationProblem}
The BRP has its interest in making as much profit as possible, because the BRP's profit is increased by utilizing as much of the energy produced locally by the BRP as possible, and only import energy when needed to ensure all EVs are minimum charged. It is only natural to formulate their goals as an optimization problem, and think about how profit can be increased using our system. The description of this optimization problem follows. All variables used below are listed in the following list and should be used as a reference when reading the functions below. The acronyms $E$ stand for energy as the first letter and export as the second in the below acronyms, $I$ for import, $U$ for used, and $P$ for price.

\begin{itemize}
  \item $n : \text{number of time slots}$
  \item $c : \text{number of charge-offers}$
  \item $y_{ij} : \text{cell } (i, j) \text{ in resulting schedule (see \Cref{tab:sched})}$
  \item $x_{ij} : \text{cell } (i, j) \text{ in charging table (see \Cref{tab:unsched})}, (x_{ij} \ass 0 \vee 1)$
  \item $\var{EI}_t : \text{energy to be imported }(\var{EI}_t \geq 0) \text{ in time slot } t$
  \item $\var{EIP}_t : \text{price for importing energy in time slot } t$
  \item $\var{EEP}_t : \text{price for exporting energy in time slot } t$
  \item $\var{EE}_t : \text{energy to be exported }(\var{EE}_t \geq 0) \text{ in time slot } t$
  \item $\var{CO}_i : \text{charge-offer } i$
  \item $\var{EU}_t : \sum_{i \ass 1}^{c} \left(x_{it} \cdot y_{it}\right)$, the total energy consumed (used) by all EVs in time slot $t$
  \item $\var{EUP}_t : \text{ price for using energy in time slot } t$
  \item $\var{EP}_t : \text{energy produced } (\var{EP}_t \geq 0) \text{ in time slot } t$ 
\end{itemize}

We define the optimization problem the schedulers need to solve as \Cref{eq:opprob}:
\begin{equation}\label{eq:opprob}
  \begin{aligned}
    & \underset{\var{EU}_t, \var{EE_t}, \var{EI_t}}{\operatorname{maximize}} & & \sum_{t \ass 1}^n \left(\var{EU}_t \cdot \var{EUP}_t + \var{EE}_t \cdot \var{EEP}_t - \var{EI}_t \cdot \var{EIP}_t\right) \\
    & \operatorname{subject\;to} & & t \ass 1, \dots, n : \var{EU}_t+\var{EE}_t = \var{EI}_t + \var{EP}_t \\
    & & & t \ass 1, \dots, n : 0 < \var{EEP}_t < \var{EUP}_t < \var{EIP}_t \\
    & & &   m \ass 1, \dots, c : \var{CO}_m\var{.minEnergy} \leq  \sum_{i \ass 1}^{n} \left( y_{mi} \cdot x_{mi} \right) \leq \var{CO}_m\var{.maxEnergy}
  \end{aligned}
\end{equation}

Where $y_{ij}$, $\var{EI}_t$, and $\var{EE}_t$ are the unknown variables we wish to solve for. The subscript $i$ has a limit constrained to the amount of charge-offers $c$, and the subscript $j$ runs to the amount of time slots, $n$. This results in there being $n \cdot m$ variables for the two dimensional variable $y$. $\var{EI}_t$ is an unknown variable denoting how much energy should be imported in time slot $t$. There are a total of $n$ time slots, resulting in $n$ variables for \var{EI}. Lastly $\var{EE}_t$ represents the amount of energy to be exported for time slot $t$. As with the variable representing energy imported, there are $n$ variables for \var{EE}.

To maximize profits, \var{EUP}, \var{EIP}, and \var{EEP} need to be set to appropriate prices, because they define the money the BRP receives from selling energy to consumers and exporting to the market, respectfully.

The core of the optimization problem is the objective function. We are interested in maximizing the energy used per time slot, taking the electricity import and export prices into account. It is therefore essential to express to only import in time slots where the import price (defined below by \var{EIP} -- energy import price) is at its lowest. The BRP should therefore only export in a time slot if it is not economically viable to send it to their consumers, or if there are simply not enough consumers at that given time slot, as we define \var{EEP} to be less than \var{EUP}, i.e.\ it is always more profitable to charge EVs than to export energy to the grid. The objective function is formally defined as:

\[
  \underset{\var{EU}_t, \var{EE}_t, \var{EI}_t}{\operatorname{maximize}} \sum_{t \ass 1}^n \left(\var{EU}_t \cdot \var{EUP}_t + \var{EE}_t \cdot \var{EEP}_t - \var{EI}_t \cdot \var{EIP}_t\right)
\]

Energy needs to be balanced in all time slots, as described in \Cref{sec:marketmodel}. I.e.\ if you need to consume more than you produce, importing energy is required. If you somehow cannot consume every bit of energy you produce to charge EVs, you have to export. This constraint is formally defined as:

\[
  t \ass 1, \dots, n : \var{EU}_t + \var{EE}_t = \var{EI}_t + \var{EP}_t
\]

For the optimization to only import energy when needed to ensure that all EVs are minimum charged, and only export if all EVs cannot be charged more, another constraint is needed. The price of consuming energy locally needs to be more than what you can get for exporting it, but less expensive than the price of importing electricity. This constraint is formally defined as:

\[
   t \ass 1, \dots, n : 0 < \var{EEP}_t < \var{EUP}_t < \var{EIP}_t
\]

Finally we need to ensure all EVs are charged in a range between their minimum energy needed to drive the range specified by the user and the maximum energy the battery can hold. This last constraint is formally defined as:

\[
   m \ass 1, \dots, c : \var{CO}_m\var{.minEnergy} \leq \sum_{i \ass 1}^{n} \left( y_{mi} \cdot x_{mi} \right) \leq \var{CO}_m\var{.maxEnergy}
\]

Besides ensuring that all EVs are charged within their range of minimum and maximum energy, this constraint also ensures that no EV can charge after its deadline. Recall \Cref{tab:sched}, where ``\charge{}'' indicates when an EV can charge and ``\nocharge{}'' indicates when an EV cannot charge. When we reach a time slot where an EV cannot charge, i.e.\ when we read a ``\nocharge{}'', $y_{mi} \cdot x_{mi}$ will result in $0$ because $x_{mi}$ will be $0$. Thus $\var{EU}_t$ for that time slot will also equal $0$, which results in the desired outcome of no EV being charged after its deadline.

This sums up the optimization problem we wish to solve. Below we will introduce different approaches to solving the same problem. This is done because we wish to compare methods in \Cref{chap:evaluation}.


Below we introduce the scheduling techniques that are candidates for solving this optimization problem, and propose a design of each chosen scheduling approach based on the techniques. To benefit from the flexible consumption of energy, given EVs that are available for flexible consumption, we can plan the way that EVs are charged, such that imbalances are minimized and all EVs are \var{minCharged} by their deadlines. 

\subsection{No Scheduling}\label{sec:nosched}
While not a model, no scheduling is the baseline, control group option. The current real world approach is not to do any scheduling of consumption, which, as we have shown, has its issues. 

Charging of each EV will simply happen as soon as it is plugged in until it is plugged out or charged to the battery's limit, taking none of the data structures above into account. This method poses a problem, because during low demand periods such as during the night, wind energy production will risk going to waste, because there simply is no use for it, as the EVs will already have been charging all night. 

\subsection{Scheduling with Linear Programming}\label{sec:lpsched}
The optimization problem described in \Cref{sec:idea} is a linear programming problem. Linear programming (LP) is a mathematical optimization technique for optimization of a linear objective function, subject to linear constraints. LP finds the \emph{optimal} solution to the objective function, being global maximum or global minimum within a feasible region. LP is interesting to analyze because it guarantees to find the optimal solution and also is very relevant to the problem described above. For LP to work, a linear objective function and linear constraints are required, as the objective function is the function that will be optimized. Our objective function is repeated here for accessibility:

\begin{equation}\label{eq:lpprob}
  \mathop{f}(x) = \sum_{t \ass 1}^x \left(\var{EU}_t \cdot \var{EUP}_t + \var{EE}_t \cdot \var{EEP}_t - \var{EI}_t \cdot \var{EIP}_t\right) 
\end{equation}

The table of charge-offers and energy prices are also supplied as input, but are static variables. The parameter $x$ is the series of time slots desired to schedule for based on the static parameters. This function is linear because it satisfies the properties of additivity and homogeneity. Additivity holds because if you have 72 time slots of length \SI{20}{\minute} in a day, then $\mathop{f}(1 + 2) = \mathop{f}(1) + \mathop{f}(2)$, due to the fact that computing profit for time slots one and two together gives the same results as computing them individually. Additivity also implies homogeneity in our case.

All the constraints described above also fit the canonical form of a LP problem, again making it trivial to transfer it to a solver.  We do not concentrate on the direct implementation of a LP solution, since many different kinds of solvers exist as off the shelf products. To make the optimization problem compatible with our idea of scheduling, the LP scheduler would work in the following three basic steps: 

\begin{enumerate}
  \item Generate input table of charge-offers and maximum charging speeds like \Cref{tab:unsched}
  \item With the input table as input, run an LP-solver that solves \Cref{eq:lpprob}
  \item Interpret output of LP-solver as a charge-plan
\end{enumerate}

The use of LP ensures that the requirements of the charge-offers are met, e.g.\ all EVs receive the energy they need before their deadlines. It also maximizes the amount of RES we use, charging as many EVs as possible if we have a lot of production. Meanwhile it minimizes energy imports, if energy imports are needed the energy will be imported in the time slots that are cheapest, all whilst taking every EV's deadline into account.

\subsubsection{Zero-one Integer Problems}
The problem solved by the objective function in \Cref{eq:opprob} has an unknown binary variable. This means the nature of the optimization problem is a zero-one integer linear programming problem. Solving zero-one integer linear problems is much more expensive in time than solving ordinary linear problems, where integers are considered as continuous. In fact, zero-one integer programming problems have been proven to belong to the complexity class NP~\cite{Karp72}.

We can therefore conclude that the optimization problem described in \Cref{eq:opprob} is NP-complete, getting the optimal solution to the scheduling problem is not realistic. However, a LP-solver can still be used to aid us in solving the problem. In order to speed up the solving of our mixed integer linear programming problem, the binary constraint can be changed to be a continuous constraint, meaning that the unknown binary variable can instead be in the range between $0$ and $1$. Then the values of the intermediate matrix will have to be rounded to either zero or one, while ensuring the constraints are held. When rounding, it can no longer be ensured that the given output is an optimal solution, but it reduces the problem to a normal linear programming problem which are much easier to solve, i.e.\ complexity class P~\cite{LPMS}. 

\subsubsection{Linear Programming Complexity}
Following the steps in the above list, there technically exists three completely different time complexities relating to the input table, running the solver, and translating the output of that usable charge-plan back to something the system can comprehend. Step 1 and 3 are straightforward, as they simply iterate over all charge-offers for each time slot. 

The dominating step of the three steps is Step 2, since Step 2 is to solve a zero-one integer problem. Step 2 is in the complexity class NP, and its input size is the number of charge-offers times the number of time slots. Since the complexity of calling a LP tool is far greater than generating the input and reading back the output, the total worst case complexity of the LP scheduler is equal to the worst case complexity of the external tool.

\subsection{Greedy Scheduling}\label{sec:greedyschedanal}
Greedy algorithms are often a quick way to solve problems, so greedy scheduling would be an interesting approach to schedule generation. However, because it is greedy, it does not guarantee that the result is optimal. We can implement a greedy scheduler in several ways, such as shortest charge first, longest charge first, latest deadline first or earliest deadline first. To get the best result, we should schedule with regard to earliest deadline first. This approach will give the best result, as the EVs with the earliest deadlines must finish before the others, and if we did not prioritize these first, we would end up with less flexibility than what we started with, because the time between the current time and the deadline will decrease by each schedule. The first thing the greedy scheduler should do is thus to sort the list of charge-offers with regards with deadlines. 

% Thalley - Ikke nødvendigt IMO
%A greedy scheduling algorithm can be implemented in several ways, but some of these way does not work properly. If we have a number of EVs to charge, we have start, end and execution time of those charges. A greedy approach to schedule EVs with regards to deadlines and charging is much like scheduling to minimize lateness~\cite[p.~8]{GREEDY}. We can thus schedule regarding to the end, start and execution time variables. However, we will show that some of these are not feasible.

%\begin{description}
%  \item [Smallest duration first:] This approach sorts by charging duration (in our case, how much time it takes to charge the EVs). This approach fails because it cannot take deadlines into account and will not generate a feasible schedule respecting the constraints.
%  \item [Smallest value first:] This approach does not give an optimal solution either. If for example an EV has to be charged for a long time, it should be schedule first, but with smallest value first, it will be scheduled late and it might not be able to charge before its deadline.
%  \todo{Hvad menes der med value? Kan ikke forstå det ud fra teksten\ldots -- Martin}
%  \item [Largest value first:] For the same reason as above, this approach is not feasible either. Here short charges with short deadlines might not be scheduled in time.
%  \item [Shortest deadline first:] Contrary to the previously mentioned approaches, this approach \emph{is} feasible. If we schedule regarding to the deadline, we get an optimal schedule. To schedule based on deadlines, we first sort the list of activities to be scheduled by their deadline. Then we simply just insert the activities into the schedule from start to end, if they fit regarding to the constraints. The proof that we get an optimal schedule can be seen in \cite[p.~8]{GREEDY}.
%\end{description}


\subsubsection{Design of a Greedy Scheduler}
To see how a greedy scheduler could possibly solve the optimization problem effectively, we now design a greedy scheduler that does not guarantee an optimal solution, but could possibly be faster than a linear programming approach. This model will be described as \Cref{alg:greedy} first and then the algorithm will be illustrated in \Cref{fig:greedy}. This algorithm works in three phases:

\begin{enumerate}
    \item Schedule all charge-offers greedily without using more energy than what is produced (\Cref{fig:greedyp1})
    \item If any charge-offer is not \var{minCharged} after first phase, use energy from the grid to schedule these. These charge-offers must be scheduled when the energy is cheapest to import (\Cref{fig:greedyp2}).
    \item If we have more power left, schedule charge-offers that are not \var{maxCharged} with the remaining energy (\Cref{fig:greedyp3})
\end{enumerate}

\Cref{alg:greedy} below follows the above steps and ensures that all EVs, represented by charge-offers, will be charged with the minimum energy amount needed, but it does not optimize such that we do not use energy from the grid. 

\begin{algorithm}[!htb]
\small
\caption{Greedy scheduler}\label{alg:greedy}
	\begin{algorithmic}[1]
		\State Input: \var{COs} = List of charge-offers, \var{Prod} = List of production values for each time slot, \var{Prices} = List of prices for each time slot
		\State Sort(COs).byDeadline
		\ForAll {\var{values} in \var{Prod}}
		  \While{\var{powerUsed} $<$ \var{production}} \Comment{Phase 1}
			\State Schedule charge-offers that are not charging from start to end
		  \EndWhile
		  \While{All charge-offers are not \var{minCharged}}  \Comment{Phase 2}
			\State Find the cheapest time slot
			\State Schedule charge-offers that are not charging and have deadline after current time slot
		    \State Set cheapest time slot price to \num{9999} (so it will not be used again)
		  \EndWhile
		  \While{\var{powerUsed} $<$ \var{production}} \Comment {Phase 3}
			\State Schedule charge-offers that are not \var{maxCharged} and are not charging
		  \EndWhile 
		\EndFor
		\State \Return List of charge-offers to be charged in next time slot
	\end{algorithmic}
\end{algorithm}

\begin{figure}[!htb]
  \centering
  \subbottom[\emph{Phase 1}\label{fig:greedyp1}] {%
    \includegraphics[width=0.8\textwidth]{drawings/greedyPhase1.tikz}
  }
  \subbottom[\emph{Phase 2}\label{fig:greedyp2}] {%
    \includegraphics[width=0.8\textwidth]{drawings/greedyPhase2.tikz}
  }
  \subbottom[\emph{Phase 3}\label{fig:greedyp3}] {%
    \includegraphics[width=0.8\textwidth]{drawings/greedyPhase3.tikz}
  }
  \caption{Plots over the different phases in the greedy scheduler algorithm defined in~\Cref{alg:greedy}. Time is on the $x$-axis and power is on $y$-axis. Phase 1 schedules all charge-offers without using more energy than what is produced from RES. Phase 2 schedules charge-offers not \var{minCharged} when energy is cheapest. Phase 3 schedules charge-offers to use remaining energy.}\label{fig:greedy}
\end{figure}

\subsubsection{Algorithm Analysis}
The time complexity of \Cref{alg:greedy} is described in \Cref{tab:greedyanal}. Lines 4 to 14 are repeated for each time slot. Lines 7-11 require $p \cdot n^2$ steps because finding the cheapest time slot takes $p$ steps, and going through each charge-offer to check if it charging takes $n \cdot n$ time, and is also the reason why lines 12-14 require $n \cdot n$ steps, assuming that we keep a list of which charge-offers that are charging in each time slot. \Cref{tab:greedyanal} shows the time complexity of the greedy scheduler. 

\begin{table}[!htb]
	\centering
	\begin{tabular}{l l}\toprule
		Line(s) & Complexity \\ \midrule
		2       & \bigo{n \log n} \\
		3       & \bigo{p} \\
    4--6    & \bigo{n} \\
		7--11   & \bigo{p \cdot n^2} \\
		12--14  & \bigo{n^2} \\ \midrule
    Total   & \bigo{p \cdot (p \cdot n^2)} \\ \bottomrule
	\end{tabular}
	\caption{Time complexity of \Cref{alg:greedy}, where $n = |\var{COs}|$, the number of charge-offers, $p = |\var{Prod}|$, the amount of time slots.}\label{tab:greedyanal}
\end{table}

\FloatBarrier

\subsubsection{Randomized Greedy Search}\label{sec:randomsearch}
An alternative approach is to use randomized greedy search as a scheduling method. \textsc{mirabel} used two algorithms to solve the scheduling problem: \emph{randomized greedy search} and \emph{evolutionary algorithm}. The randomized greedy search is simply a randomized greedy algorithm which randomly chooses flex-offers, constructs schedules and analyzes which schedule is best. The evolutionary algorithm takes the schedules from the randomized greedy search, selects some of the solutions, then uses crossover and mutation to change and optimize the schedules. Both of these algorithms are time limited, e.g.\ ``run for 20 seconds and return result''. This is scales badly and will therefore not be considered.


\subsection{Discussion}
An LP solution approach is the optimal approach to solving such an optimization problem. We however choose to implement the LP solution with rounding, using software tool designed to solve linear optimization problems. This approach will be more time efficient, but it cannot be guaranteed that the output is optimal. We choose to also implemented a greedy scheduler so that we can compare the results of different scheduling methods with the results of no scheduling. We decide not to design a randomized greedy scheduler, because it does not guarantee a good solution. We will not implement any other solutions because we will not have time to implement them properly. 

We will evaluate these methods with regards to the quality of the scheduling and with regards to their performance in \Cref{chap:evaluation}.
