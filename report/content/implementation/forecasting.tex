\section{Price Forecasting}\label{sec:priceforecasting}
As described in \Cref{sec:timeseriesforecasting}, we decide to forecast prices of electricity as way to increase profit for the BRP. To save time during development, we implement the open-source machine learning algorithm collection for Java called \href{http://www.cs.waikato.ac.nz/ml/weka/}{Weka}, which offers many different approaches to machine learning, among other areas. We use it for data preprocessing and for building the models for forecasting. Weka does not include any methods to model a time-dependent series of data points. Pentaho, an open-source business intelligence software developer, have written a plug-in for Weka that eases time series forecasting, which we use to help predict electricity prices~\cite{pentahoTSF}.

By using Weka and the time series forecasting API in the system, we achieve a structure that allows interchanging of forecasting types on our data. This data consists of electricity prices for every hour-interval (\hour{0} -- \hour{1}, \hour{1} -- \hour{2}, etc.) of every day from \protect\formatdate{1}{1}{2011} until \protect\formatdate{11}{11}{2013}. As shown in \Cref{sec:timeseriesforecasting}, an hour-interval of this data looks linear, but as we show in \Cref{fig:realEnergyPrices} in \Cref{sec:forecasteval}, the linearity is questionable. We therefore use Weka's ability to try different forecasting methods, to see how accurately they forecast. See \Cref{sec:forecasteval} for an evaluation of these methods.

We build a price forecasting module on top of Weka, that allows looking up prices in the database and forecasting prices past the edge of this database using selectable forecasting methods. This allows monitoring error percentage and comparison of the methods and more freedom when scheduling charge-offers. It is possible to choose between linear regression, Gaussian process, and multilayer perceptron models in the module, all of which are implemented by the time series forecasting plug-in for Weka. The interface also allows modifying what percentage of data is part of the test and evaluation sets. We use the forecasted prices for the scheduler implemented in the system.
