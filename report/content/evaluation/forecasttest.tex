\section{Forecasting Evaluation}\label{sec:forecasteval}
In this section, we test the applicability of the system's price forecasting models and compare their MAPE (Mean Absolute Percentage Error) values to determine which method is best. We have adopted MAPE as the accuracy criteria for the tests in this section. The testing of the forecasting of prices were independently forecasted one day ahead for 14 days, i.e.\ after forecasting a day, the historical data was appended to the learning data of the forecasting model, and the next day was forecasted, and so on to keep the model as updated as possible. Weka, the tool used to forecast, chose the test and evaluation set split, resulting in \perc{66} of the data used for the test set and the rest used as the evaluation set.

\subsection{Test Data}
The various forecasting models are built on instances of electricity price data collected every hour of each day from \protect\formatdate{1}{1}{2011} and onwards to the day before forecasting. \Cref{fig:realEnergyPrices} plots the circa \num{25000} prices from the first month of 2011 until \protect\formatdate{11}{11}{2013}. This data is retrieved from the Nord Pool Spot electrical energy market, owned by Nordic and Baltic transmission system operators operating in Scandinavia, Estonia, and Lithuania~\cite{nordPoolSpotTSOs}. Prices used are \emph{system prices}, meaning the reference price of electricity computed Nord Pool Spot by disregarding any bottlenecks in the market and then computing the average price~\cite{nordpoolSystem}. This price therefore covers the entire market instead of the median price traded for a given hour of a day in an area, such as an area in Denmark or Sweden. This results in a more generic market price overview, not taking any specific region's extremes into account.

\begin{figure}[htpb]
  \centering
  \includegraphics[width=\textwidth]{drawings/historicalEnergyPricesData.tikz}
  \caption{Test data consisting of electricity prices between \protect\formatdate{1}{1}{2011} \hour{0} and \protect\formatdate{11}{11}{2013} \hour{23}}\label{fig:realEnergyPrices}
\end{figure}

\subsection{Forecasting Results}
With the data from \Cref{fig:realEnergyPrices}, the results of testing the correctness of our forecasting models across 14 days can be seen in \Cref{fig:priceerrorTest}. Forecasting is done as stated above, with the dataset being updated after every day of forecasting.

\begin{figure}[!htb]
  \centering
  \includegraphics[width=\textwidth]{drawings/pricemape/priceerror.tikz}
  \caption{Actual hourly price and forecasted prices for the two forecasting models between \protect\formatdate{12}{11}{2013} \hour{0} and \protect\formatdate{26}{11}{2013} \hour{23}}\label{fig:priceerrorTest}
\end{figure}

Note that with the multilayer perceptron model, forecasted prices for the 13\textsuperscript{th} and between the 18\textsuperscript{th} and 19\textsuperscript{th} drop well below the $x$-axis, causing this model to be very unpredictable and thus have a high MAPE value, as shown next. Besides the offset forecasts of the multilayer perceptron, it is still clear from \Cref{fig:priceerrorTest} that this model is far from accurate. Linear regression on the other hand seems to neatly follow the local maxima and minima of the actual, historical prices, albeit a slight offset in price.

To formally measure how effective the two forecasting models are, we compute the MAPE value for each based off of the actual historical price and the forecasted price of energy for the forecasted value for each hour between the days in \Cref{fig:priceerrorTest}. This is done by updating the model to the day before forecasting, saving the forecast, then waiting a day to compare this forecasted data with the actual prices. MAPE in percentage is defined by the formula:

\[
	M = \frac{1}{n} \sum\limits_{t=1}^n \left\lvert\frac{A_t - F_t}{A_t}\right\rvert \cdot 100
\]

where $n$ is the amount of time intervals, $A_t$ is the actual value, and $F_t$ is the forecasted value. If the MAPE value is zero, a perfect fit exists and there are no errors. So the higher the MAPE value, the less attractive the model is.

The MAPE value of price forecasting with linear regression evaluates to \perc{9.28}, and \perc{15.54} for multilayer perceptron. These value are unfortunately relatively large compared to what we had expected, and following are some guesses as to why.

Electricity prices are difficult to predict because they are based on many factors. Weather, for example, is quite unpredictable and it determines the need for heating devices and air conditioning devices. It also determines how much power is produced with cheaper renewable energy production and how much is produced using expensive gas and coal works. Fuel prices also affect electricity prices, since most electricity is generated through a power plant that uses oil or gas, affecting the prices~\cite{nordpoolSpotProducers}. The data used for testing is also only based on the past 2 years, which can make the prediction unstable. Inflation begins to play an important role in prices if using much older data, resulting in more complications if we do not take inflation into account. We can conclude that entirely basing electricity pricing on historical data is not enough, since the nature of the forecast depends on many factors. That said, having this information is better than nothing, however it could be greatly improved, factoring in all the mentioned parameters.
