\section{Discussion}\label{sec:discussion}
We will in this section discuss the results from our tests and evaluate our system as a whole. We will start by discussing the results from our tests. 

\subsection{Testing the System}
We tested how profit was influenced by different cases to see which factors had the most influence, but also to see how much we can increase the profits for the BRPs. We did this by using three cases with unique standard parameters. This, however, only gave us a very limited amount of data points to evaluate profit increases on and is merely a indication of how our system may improve profits. To get better test results we should have tested with more cases, but also with entirely different days. All the test cases were tested with the same amount of EVs on the same day (\num{700} and \protect\formatdate{2}{5}{2013}, respectively). Using different days and other amounts of EVs might have provided different results, which might have been better or worse. If we had changed the costs of each type of energy import/export, we would have seen a very different result. Another thing that we should have tested is how sizes of time slots affect the profit. We can only assume that smaller time slots would generate better schedules and thus increase profits, but we have no way to proove this with our current results. 

We only tested two different scheduling methods and compared these to no scheduling. Neither of the two methods we use are optimal and will thus not increase profits as much as it could be. To get the best results, we should at minimum have tested using an optimal scheduler as well and perhaps some entirely different methods, to have a better comparison and to find the truly best scheduling method. We only tested using two schedulers because of the time limit this project. 

\subsection{Evaluation of the System}
The results we have from the tests are very pleasing. An increase of \perc{13} might seem like a small increase, but in a large business like this, a profit increase of \perc{13} is a huge deal. We, however, believe that the number is actually a lot larger than that. This is because the price we use to calculate profits with, is the price for importing or exporting energy and is in fact lower than the price consumers pay for energy~\cite{energyPricesCon}~\cite{energyPricesBRP}. We had no way of truly testing the profit increase, but we are sure that a BRP would gain a lot from using our system, especially since it would be easy for the BRP to actually use it. They would just have to sell intelligent charging stations to their consumers (which they might even gain from as well) and then simply run the system on a computer. 

Regarding the performance tests, we are very pleased with the efficiency of the greedy scheduler. We mentioned in \Cref{sec:problem_statement} that Denmark would have \num{281621} EVs by 2020 if there was a charging station availability of \perc{90}. In \Cref{sec:evalperformance} we showed that the greedy scheduler was capable of handling up \num{268000} EVs within a time slot of 20 minutes. In that manner, a single BRP would be able to schedule almost all EVs with time slots of 20 minutes on a household computer system. This is a lot more than what is actually needed, as we have multiple BRPs in Denmark. This means that we could reduce the size of time slots and actually receive better scheduling. The LP scheduler was however a lot slower. This is not necessarily because of our design nor the problem. We used the GLPK as a linear programming solver, which is actually a quite slow and poor performing LP solver \cite{LPBENCHMARK}. Had we chosen some other LP solver, we might have gotten better results in both performance and in profits, which might make the LP scheduler just as viable or better than the greedy scheduler. Although this may help, it does not change the fact that a zero-one linear programming problem is NP-complete.

Our forecasting model has a best case MAPE value of \perc{9.2} which is not bad, but it is not very good either. We think that the tool used to achieve the highest MAPE value, Weka with linear regression, may not be basing its forecasts off the previous years' worth of prices for each hour-interval, but building its model on price data for entire days. However from our test cases, we see that price forecasting errors does not have a huge influence on profits, but in a case where we would have to import a lot of energy, we assume that it would have a large influence, and we should thus improve this. We never tested how well the weather forecasts are, but as we use an external system for weather forecasts, it is likely to change depending on the BRP. Better forecasts could be gained from making a deal with a local weather station.

Lastly we have the solution model itself. We wanted to schedule every 20th minute and charge for 20 minutes. We could have made the problem a lot more flexible in the two following ways. First could be to allow EVs to charge for fractions of time slots (e.g.\ 7 minutes) rather than whole time slots. This would mean that we would be able to schedule exactly according to the forecasted production but also to schedule such that only charge what is minimum needed at all times (depending on the situation of course). Second, you could allow EVs to charge for fractions of their maximum charging speed. If we have a production of \SI{18}{\kilo\watt} and two EVs with a charging speed of \SI{10}{\kilo\watt} each, we would only be able to charge one of them without going over the production. If we could charge one of the EVs with \perc{100} of its charging speed and the other with \perc{80} of its maximum charging speed, we would completely balance production and consumption. The reason why neither of these two ways were our solution model is because they seemed a lot more complex and harder to solve than with fixed time slot sizes and charging speeds. After we have worked with the problem, we definitely believe it would be possible to add this extra flexibility and get better overall results. 
