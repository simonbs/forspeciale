\section{Project Results}\label{sec:results}
This section describes the final results of our project. 
First we will discuss whether our system meets the requirements, 
setup in \Cref{sec:requirements-specification}. 
Then we will conclude the project based on our problem statement from \Cref{sec:researchstatement}.

\subsection{Requirements}\label{sec:conclusion-requirements}
\subsubsection{Functional Requirements}
\begin{description}
  \item[\yes Add smart devices] The adding of smart devices is done by connecting them to the smart hub (HomePort). If any device is added to HomePort, they are visible in our system. 
  \item[\no Assign a location to devices] It is not possible to assign locations to devices in our application. We have, however, made a design for it. 
  \item[\yes Create gestures] It is possible to create new gestures.
  \item[\yes Train gestures] It is possible to train gestures, but only when creating it. 
  \item[\yes Recognize gestures] We can recognize gestures using the \$3 Gesture Recognizer.
  \item[\yes Assign gestures to actions] We can add gestures to actions. Multiple actions can share same gesture.  
  \item[\yes Control smart devices by pointing at them] We can control smart devices by pointing at them.
\end{description}

\subsubsection{Performance Requirements}
\begin{description}
  \item[\yes Realtime gesture recognition] The requirement was to recognize gestures faster than \SI{200}{\milli\second} per gesture. We are able to do this up to a certain number of gestures. We measured less than \SI{120}{\milli\second} for \num{20} unique gestures, trained \num{5} times each, using an iPhone 5.
  \item[\yes Scalability] The system shows linear scalability in terms of performance in gesture recognition. 
  \item[\yes Correct gesture recognition] The requirement was \perc{80} correctness rate for gestures. The \$3 Gesture Recognition paper \cite{threedollar} reports a correct recognition rate of \perc{80}.
  \item[\no Correct device selection] The requirement was to send the action to the correct device \perc{80} of the time. Based on the average mean distance error of \SI{2.92}{\meter} that we have measured, we get a correctness rate of just \perc{4.29}. 
\end{description}

\subsubsection{Overall Requirements}
\begin{description}
  \item[\yes Use existing hardware] The system is implemented using only existing hardware. The only non-commercial hardware is the smart hub. 
  \item[\yes Not limited to line of sight] We are not limited to line of sight. 
\end{description}

\subsection{Results}
The problem statement in \Cref{sec:researchstatement} asks:
\begin{framed}
  How can wearables be utilized for home automation in a gesture driven solution?
\end{framed}

We have developed a system that uses a wearable device as a control for home automation. 
The system works by recognizing \emph{three-dimensional gestures}, 
created by moving the wearable around, 
and making a gesture trace by measuring accelerometer data. 
Each recognized gesture corresponds to an \emph{action}, 
which is carried out by the device the user is pointing at with the wearable device. 
To figure out which device we point at, 
we use the magnetometer in the device for orientation, 
and Bluetooth Low Energy (BLE) beacons for indoor location. 
The system can calculate which device that is being pointed at, 
using the user's location and orientation, 
as well as the smart devices' locations 

In our implementation we switched out the wearable for a phone, 
as this made the implementation and testing easier. 
\Cref{sec:conclusion-requirements} discussed whether or not the system met the requirements we had set for it. 
While most of the requirements are met, 
one of the most important ones,
was not met. 
Our results showed that the mean distance error we could get from the BLE beacons was \SI{2.92}{\meter},
which resulted in a correctness rate of pointing of just \perc{4.29}. 
This very low rate is unacceptable for the system that we wanted, 
and we can conclude that BLE beacons \emph{cannot} provide the indoor location accuracy that we need. 
Our tests show that we need an accuracy of around \SI{0.5}{\meter} to be viable. 
