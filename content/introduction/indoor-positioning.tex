\section{Indoor Positioning}\label{sec:indoor-positioning}
\todo[author=Thalley]{Make indoor positioning section fit in better}
%Something about Estimote
In order to determine which device the user points at and thereby intends to control, 
it is necessary to determine the locations of the devices in the system relative to the user.

The focus of this project is not to position devices and as such it was not the intention to spend time developing an entire solution for positioning devices indoors. 
Instead it was desired to find an existing product that could be used to facilitate indoor positioning.
The solution should be available in the early phases of the project in order to start building the system based on the solution for positioning.

Ideally users of this project should be able to control any device that fits within the concept of Internet of Things he owns, 
the price for any device needed to position each controllable device should be low. 
If a user owns several devices that can be controlled using gestures and an extra device is needed for each in order to preform the positioning, 
the price of such a device should be at a minimum.

It is assumed that users already own one or more devices that fit within the concept of Internet of Things and possibly are early adapters of such technology, 
it is assumed they have some technological expertise. 
However, it easy to imagine that this project can be used in an office environment where employees of varying technological expertise work or in health care. 
Therefore users may have a varying degree of technological expertise and it should be easy to extend the solution with new controllable devices.

Naturally the accuracy of the solution used for positioning objects plays an important part. 
\Cref{fig:indoor-positioning:incorrect} shows the consequence of an incorrect location. 
If a lamp is estimated to be at another location that it is actually located, 
the user must point to an incorrect location in order to control the lamp.
Furthermore if the estimate is too wide, that is, the given area in which the lamp is located is very big, 
there is a greater risk that locations overlap. 
Overlapping locations causes a complexity as it is necessary to determine which device the user desires to control if he points at the overlap as visualized in figure \ref{fig:indoor-positioning:overlap}.

\begin{figure}[!htb]
    \centering
    \begin{minipage}[t]{0.45\textwidth}
    \centering
    \includegraphics[width=0.6\textwidth]{images/incorrect-positioning-estimate.png}
    \caption{Incorrect location estimate. The estimate is visualized as a striped circle.}
    \label{fig:indoor-positioning:incorrect}
    \end{minipage}\qquad
    \begin{minipage}[t]{0.45\textwidth}
    \centering
    \includegraphics[width=0.6\textwidth]{images/positioning-overlap.png}
    \caption{Overlap of estimated positions. The estimates are visualized as a striped circle.}
    \label{fig:indoor-positioning:overlap}
    \end{minipage}
\end{figure}

Based on the above the following criteria for assessing potential solutions can be outlined.

\begin{itemize}
    \item Availability
    \item Price
    \item Ease of use
    \item Accuracy
\end{itemize}

Only solutions intended for indoor positioning was considered and thus GPS is not considered a potential solution. 
GPS is meant for outdoor positioning and a signal is not always available while indoors and even if it is, the accuracy of the estimated location is very low.

\begin{table}[h]
    \centering
    \caption{Assessment of potential solutions for indoor positioning. Please not that all prices are converted to U.S. dollars from their respective currency. Prices include the minimum available hardware for positioning a device.}
    \label{tbl:indoor-positioning}
    
    \begin{tabularx}{\textwidth}{XXXXX}
        \textbf{Product} & \textbf{Availability} & \textbf{Price} & \textbf{Ease of use} & \textbf{Accuracy} \\
        
        Estimote Beacons and Stickers \cite{estimote}
        & Beacons and Stickers are shipping. SDKs available.
        & \$99 for beacons. \$99 for 10 stickers, one per device to be positioned.
        & Initial installation of beacons. Attach each sticker to device.
        & Unknown. Desired to be less than five meters. \todo[author=Simon]{Update after conducting tests.} \\
        
        Pozyx \cite{pozyx}
        & Available for preorder.
        & \$368 for anchors. \$123 for each device to be positioned, plus supported Arduino.
        & Initial installation of anchors. One tag for each device, plus supported Arduino. Not meant for mounting.
        & Claimed to be 10 cm. Untested.
        
    \end{tabularx}
\end{table}