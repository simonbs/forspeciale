\section{Problem Statement}\label{sec:researchstatement}
This chapter has explored different wearables, 
the components commonly found in wearables and different levels of home automation. 
It is our hypothesis that wearables can be integrated into home automation environments, 
in order to provide an attractive interface for controlling devices connected to the Internet, 
such as light bulbs, thermostats, and music centers.

While introducing a wearable in a system provides interesting possibilities for autonomous systems, 
that adjust the environment based on the users properties, 
\eg his body temperature or the amount of light he is exposed to,
the remainder of this report will focus on the possibilities of gesture driven, \ie interactive, systems using a wearable.

The report will answer the following question:

\begin{framed}
      How can wearables be utilized for home automation in a gesture driven solution?
\end{framed}

\subsection{Scenario}\label{sec:scenario}
We imagine a scenario of a user that owns a wearable and a home with a number of smart devices. 
All of these smart devices are connected to a smart hub, 
giving him centralized control of all his devices using \eg a smartphone application.
To perform a certain action on a certain devices, the user needs to:
\begin{enumerate}
  \item Take up his smartphone
  \item Open up his smart hub application
  \item Find and select the device
  \item Find and select the action
\end{enumerate}

We envision an alternative solution, 
where the user can use his wearable to control his smart devices. 
More specifically, we want to create a smart system, 
where the user can use his wearable to \emph{point and select} a smart device, 
and then by performing a certain \emph{gesture}, he can send a certain \emph{action} to the smart device
This way there is no overhead of finding the device on an application, 
nor having to find the action. 
Thus to perform a certain action on a certain devices, the user needs to:
\begin{enumerate}
  \item Point at the device
  \item Perform a gesture
\end{enumerate}

To recognize the gesture, the arm on which the users wears the wearable must perform the action. 
The wearable itself will have to be able to track 3D movements, for example using an accelerometer,
and know which device the user is pointing at. 

%%% Local Variables:
%%% mode: latex
%%% TeX-master: "../../master"
%%% End:
