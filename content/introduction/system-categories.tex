\section{System Categories}\label{sec:system-categories}

The scenarios in which home automation is facilitated can in general be divided into the following three categories.

\begin{itemize}
\item Rule driven systems
\item Gesture driven systems
\item Autonomous systems
\end{itemize}

The categories varies in the way users configure and interact with the systems. 
``Manual systems'' could constitute a fourth category, consisting of regular systems with manual switches. 
The category was left out as it such systems do not contribute to home automation.
Each of the four categories and their use cases are briefly described below.

\subsection{Rule Driven Systems}

A system is considered to be rule driven if an action is considered when a set of rules are  fulfilled. 
The systems use conditional statements to express input and output. Below are a few examples of rules.

\begin{itemize}
\item (The temperature is above 23 degrees Celsius) $\rightarrow$ (Lower the temperature on my thermostat)
\item (The CO$_2$ index is critically high) $\rightarrow$ (Open my windows)
\end{itemize}

The above rules consist of a condition on the left-hand side of the arrow and an action on the right-hand side of the arrow.

The automation of the home is based on the set of rules. 
To achieve the desired behavior, the user must add, remove or tweak existing rules.

Examples of rule drive systems include Apples HomeKit. 
As outlined in the framework reference for HomeKit \cite{applehomekitref}, the system is based on actions and events. 
Triggers constitutes rules by encapsulating actions and events. 
Each event represent condition. An event may be fulfilled by a change in time, the state of devices in the system of the location of the user.

\subsection{Gesture Driven Systems}

Systems are gesture driven if the system is configured with a set of gestures that can be performed by the user in oder to trigger some action. 
Each device in the system responds to a set of gestures. 
For example, a lamp may respond to the user waving in order to turn on and the user clapping in order to turn off.

A gesture driven system is partly rule driven as each gesture registered in the system is associated with one or more actions. 
The association means that each time the gesture is registered in the system, the action should be triggered. 
Such rules can be formulated as ``if I wave then lower the temperature on my thermostat''

Examples of gesture driven systems includes Hiris, a wearable computer with focus on gestures that allow the users to interact with other devices \cite{hirisweb}.

\subsection{Autonomous Systems}

An autonomous system monitors the system and proactively responds to changes in the system. 
Observable changes include but are not limited to changes in the temperature, CO$_2$ index, the number of people in the room or even who are in the room.
Autonomous systems should intelligently react to the users needs based upon the observable state of the environment.

Autonomous systems rely on the concept of ambient intelligence in order to determine the necessary actions. 
Such systems include autonomous enhancement services that replaces manual care with an automated system \cite{nehmer2006living}. 
These systems gather environment and the data about the individuals body functions, 
e.g. temperature, pulse and blood pressure in order to determine if the individuals health is critical.

Autonomous systems are rule driven systems that intelligently determines the rules to be created and the necessary action to take when a set of rules are fulfilled. 
Typical for such systems may be the complexity of the rules. 
It is not given that users themselves are capable of determining a suitable set of rules in order to judge if their health is critical. 
The system should itself be able to determine such set of rules and adjust it to the individual.

Examples of autonomous systems include the one described Nehmer \etal\cite{nehmer2006living}. 
The authors envision a living assistance system which monitors elderly people. 
A model is outlined and by continuously feeding the model with data about the individuals body functions and his behaviour, 
they can determine if a critical \textit{situation} occurs. 
A critical situation could be that the person has fallen and are not responding to contact, for example calls.
Such system may reduce the cost of providing care to the elderly people.

\subsection{Conclusion}

The rule driven, gesture driven and autonomous systems are all rule driven to some extend but the origin and the types of rules differ between the systems. 
In a rule driven system the rules are configured by the user.
In an autonomous system the rules are programmed by some expert or in collaboration with experts in a certain field, for example in a medical field. 
The system may adapt its set of rules based on the environment and that behavior of the individual.
Gesture driven systems uses rules configured by the user. In such systems gestures make up the condition of a rule.

When concerned with the field of home automation it is relevant to classify each system in order to determine how automatic a system is. 
The more automatic a system is, the less the user should be involved with the system.

The degree of automation as well as the reasoning behind each of the classifications are shown in \Cref{tbl:system-categories}.

\begin{table}[h]
\centering
\label{tbl:system-categories}
\begin{tabularx}{\textwidth}{XXX}
\textbf{Gesture driven systems}       & \textbf{Rule driven systems}                             & \textbf{Autonomous systems} \\
\textit{Lowest degree of automation}  & \textit{Medium degree of automation}                     & \textit{Highest degree of automation}\\
Configured by the user.               & Configured by the user.                                  & Configured by an expert.\\
Conditions are triggered by the user. & Automatically and constantly observes the environment.   & Automatically and constantly observes the environment.\\
~                                     & The configuration may be reusable for other individuals. & Automatically adjusts to the users needs.\\
\end{tabularx}
\caption{Classification of systems based on their degree of automation}
\end{table}

%%% Local Variables:
%%% mode: latex
%%% TeX-master: "../../master"
%%% End:
