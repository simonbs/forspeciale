\section{Problem Statement}\label{sec:problem_statement}
We narrow the problem domain to a focus area based on the above introduction. The rest of the report will then concentrate on answering the problem statement defined in this section, which is based on the focus area. 

Working with EVs is interesting because the growth of EVs has increased tremendously the last couple of years. According to The Danish Transport Authority (Trafikstyrelsen)~\cite[p.~56]{TRAFIKSTYRELSEN2010}, the sales of EVs will increase by a huge margin depending on charging availability. Charging availability of \SI{100}{\percent} indicates that there is charging available within the range of all EVs. In 2010 Denmark had only \num{300} EVs. They claim that Denmark with a charging station availability of \SI{45}{\percent}, would have \num{84366} cars by 2020, and \num{281621} cars with an availability of \SI{90}{\percent}. The increased sale is due to EVs being a ``cleaner'' alternative to vehicles with an internal combustion engine and the tax discounts currently offered. This means that in a foreseeable future, estimates hint that we will have 300 times more EVs on the road after the next 7 years with an charging availability of \SI{40}{\percent} -- in Denmark alone!

We think it would be an interesting scenario to charge EVs entirely with wind energy. Seeing as EVs can be charged at night, they can be seen as flexible consumers. Balancing this flexible-demand with focus on EVs in the energy balance domain, is an area where we think we can effectively contribute with a reasonable solution, in the same methodology that \textsc{mirabel} worked with flexible-demand, however with a small scale use case of smart charging EVs. This is however an extreme situation and one should be careful with assuming too much. We will thus have to set up some conditions of the scenario we assume, which will be described in \Cref{sec:scenario}. 

Given the above introduction of the energy balancing problem, we come up with a problem statement, focusing on balancing with a flexible-demand scenario using smart grid technology, specifically with the EV as the flexible unit. This report revolves around the process of answering the following question: 
\begin{quote}
How well can we balance the production from renewable energy sources with the consumption from charging electric vehicles, using an intelligent home charging station that supports demand-response and the flexibilities in electric vehicles' power consumption?

In answering this, we will focus on the following questions:
\begin{itemize}
  \item How can we minimize imbalances?
  \item How can we maximize profit?
\end{itemize}
\end{quote}

In our case, maximizing profit and minimizing imbalances complement each other in a beneficial way. The rest of this report and our work will revolve around answering these questions. To how much of a degree we fulfill this will be followed up on in the conclusion in \Cref{chap:conclusion}.

Before we start designing a solution, we need to set up requirements for the solution. This is necessary to determine what is needed for a solution. 
