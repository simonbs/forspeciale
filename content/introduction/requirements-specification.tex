%!TeX root = ../../master.tex
\section{Requirements Specification}
\label{sec:requirements-specification}

In this section we will list the requirements that the system we develop must fulfill.
We device the specifications into functional requirements, 
\ie functions that our system must implement,
performance requirements,
\ie requirements for how well the system should perform,
and overall requirements, 
\ie requirements that our solution as a whole should fulfill.

%Thalley: Evaluerer vi 
The requirements in this section will be evaluated in \Cref{chap:conclusion}.

\subsection{Functional Requirements}
These requirements are functional requirements that our system should implement.  
\begin{description}
    %Thalley: Should we specify rooms or floors?
    \item[Setup a room] The user needs to be able to specify the positions of the Estimote beacons in a room, as well as the shape and size of the room. They should be able to do this for multiple rooms so they can use the system in their entire home.
    \item[Add smart devices] If the user is performing a first time setup or has acquired a new smart device, he needs to be able to add it to the system. 
    \item[Assign a location to devices] The system should be able to insert the position of the items without the user having to insert actual coordinates himself, \eg by an ``Assign Location'' button. This is needed first time the user adds a smart item to a room, but also when a smart item is moved or removed.
    \item[Create gestures] The user needs to be able to create new gestures and train them. This is necessary as people would not necessarily have the same preferences to gestures, but it is also easier to recognize gestures performed by the same user that created them.
    \item[Recognize gestures] The system should be able to recognize gestures in order to execute actions based on the gestures.
    \item[Assign gestures to actions] The user should be able to link a gesture to a certain action, \eg \textit{Clockwise circle} turns up the stereo.
    \item[Control smart devices by pointing at them] The system needs to detect when a user is pointing at smart items, determine which items are being pointed at and react to any gestures performed.
\end{description}

\subsection{Performance Requirements}
These requirements will be used to measure how well the system performs. 
\begin{description}
    \item[Realtime gesture recognition] The system should recognize gestures effectively to give the user a feeling of realtime control of devices.
    \item[Correct gesture recognition] The system should recognize gestures correctly at least \perc{80} of the time. If the system continuously turns the volume of your stereo up instead of down, it is rendered useless.  
    \item[Correct device selection] The system should send the action to the correct device at least \perc{80} of the time. The system needs to figure out which device is the most likely to perform the action send. The system should not turn on the coffee machine instead of turning on a lamp. 
    \item[Overall system correctness] The system should recognize the correct gesture and send the corrosponding action to the correct device at least \perc{80} of the time. This is required to give the user actual control of the devices. 
\end{description}
    
\subsection{Overall Requirements}
These requirements describe additional requirements not mentioned in the previous two requirements sections.
\begin{description}
    \item[Use existing hardware] Our system should not use any specialized or non-commercialized hardware.  
    \item[Useful throughout the day] Our system should be usable throughout the day without requiring recharging of devices. 
    \item[Not limited to line of sight] Unlike Reemo, we do not want our system to require line of sight of devices we want to control. 
\end{description}
