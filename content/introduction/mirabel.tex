\section{MIRABEL}\label{sec:MIRABEL}
\textsc{mirabel} is short for \emph{Micro-Request-Based Aggregation, Forecasting and Scheduling of Energy Demand, Supply and Distribution} and was an EU-funded project running from 2010 until 2012. The main goal of the \textsc{mirabel} project was to develop an approach to balance energy production from RES with demand from households~\cite{MIRABEL}. This project used forecasting (demand and weather forecasts) and scheduling to plan the distribution of energy. \textsc{mirabel} introduced the concept of \emph{flex-offers} which allows consumers to be more flexible with regard to their energy demands. Consumers would be able to, for example, set a deadline for when they would want their dishes to be clean by pressing a button on their dishwasher and letting a system control when it would be cheapest to turn on the dishwater, which would be when cheap energy was available. This solution decreases costs for all parties, and helps minimize the need for fossil fuels. 

\textsc{mirabel} consists of three core components that make their system attractive to explore: the flex-offer data structure, scheduling of flex-offers, and forecasting of production and consumption. Each is briefly described below.

\subsection{Flex-offers}\label{sec:flexoffer}
One of the core concepts of \textsc{mirabel} is their \emph{flex-offers}. Flex-offers are based on the fact that a lot of the energy generated from RES, changes frequently based on various factors, such as how windy or sunny it is. Sometimes these RES will generate more energy than what is consumed, and sometimes they will not generate enough energy, and the energy companies have to turn to non-RES such as coal or gas plants.

The goal of flex-offers is to use the energy generated by RES that otherwise would be wasted, for example when it is windy during the night -- the time when demand is at its lowest. For example, imagine a household with the kind of dishwasher with a button explained previously. After dinner, the dishwasher is full and the family wants the dishes clean by the morning. Instead of starting the dishwater immediately, they give it a deadline: ``Be done by \hour{7}''. Looking at the weather forecast, the system then schedules the dishwater to start when energy can be generated from wind turbines so that the energy suppliers do not have to use more expensive sources, such as coal or gas to start the dishwasher straight away. Flex-offers are thus a way of letting a system choose when to start your nonessential electric appliances to balance supply and demand, benefiting all parties and the environment. Flex-offers can be aggregated and allows the system to group different areas and make it easier for the system to scale to larger power grids with both more consumers and different types of consumers.

\subsection{Scheduling}
In \textsc{mirabel}, time intervals for scheduling are usually 15 minutes long. This means that the system finds the supply and demand for each 15 minutes and tries to balance each interval such that no interval's demand exceeds the supply. A flex-offer is scheduled \emph{only once}, and will not be rescheduled afterwards. The objective of scheduling is to minimize the cost of the BRP, who is responsible for buying and selling energy. Based on market prices, the cheapest intervals are preferred during the scheduling. \textsc{mirabel} uses a randomized greedy search with heuristics to schedule the flex-offers. For a more detailed description of \textsc{mirabel}'s scheduling, see~\cite{IS2011}. 

\subsection{Forecasting}
\textsc{mirabel}'s scheduling is based on forecasts, both for supply and demand. These kinds of forecasts are necessary in order for the system to support scheduling flex-offers. \textsc{mirabel} used two different consumption demand forecasting models: the \emph{Engle, Granger, Ramanathan, and Vahid-Araghi (EGRV) Model}~\cite{EGRV} and the \emph{Triple Seasonality Holt Winters (HWT) Model}~\cite{HWT}, both of which are too complex to be fully described in this analysis. We have no data on how they forecasted production. 

The following section explains the market models for the BRPs to analyze how BRPs earn their money and how we can improve it. 
