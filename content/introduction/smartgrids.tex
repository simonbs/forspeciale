\section{Smart Grids}\label{sec:smartgrids}
For \textsc{mirabel} to function, it requires the smart grid technology. In this section we analyze the smart grid, to gain a better understanding on what technology we can rely on when we implement a solution to the problem statement, introduced in \Cref{sec:problem_statement}.

A smart grid adds functionality to the current electric grid, where the flow of power happens between various parties connected to the grid (e.g.\ households, factories, and electricity companies), and adds an information and communication layer on top. It is an attempt to improve the current electrical infrastructure to, among many reasons, better support renewable energy sources, enable more effective maintenance, and improve resilience to disruption~\cite{jrcSmartGrid}. In order words, we can use smart grids to communicate with households. 

One practical benefit is a two-way power flow where consumers with solar panels at home, are able to generate electricity and sell it to the grid. Another technique that a smart grid can utilize, is to discharge electric vehicle batteries during peak loads, and then send this energy elsewhere in the grid. This kind of infrastructure also exploits metering demand, down to the household level, giving way to new and more accurate demand forecasting models, so that electricity production and delivery can match the consumption in a more optimal way. This is of course speculative because no consumption forecast can be \perc{100} accurate all of the time. 

With the information gathered in this and the previous sections, we will now formulate a problem statement where we will focus on a area of the problem that we wish to solve. 