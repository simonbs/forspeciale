\section{Server}\label{sec:serverimplementation}
The server was described in \Cref{sec:serverdesign}. 
In this section, we briefly document the implementation of the server.

The server is implemented in Python,
using a microframework called \texttt{Flask}\footnote{http://flask.pocoo.org/},
which provides a built-in server and a RESTful request handler. 
As described in \Cref{sec:serverdesign}, 
the server only provides two REST methods for controlling a devices and getting the devices. 
Using \texttt{Flask}, the structure of the implementation can be seen in \Cref{lst:server}.

\begin{listing}
  \begin{minted}[breaklines, autogobble]{python}
    from flask import Flask
    app = Flask(__name__)
    
    @app.route('/', methods=['POST'])
    def do_action():
      action = request.form['action']
      id = request.form['id']
      # Code to send request to HomePort
      
    @app.route('/devices', methods=['GET'])
    def get_devices():
      # Code to return devices from HomePort
    
    if __name__ == '__main__':
    app.run(host="localhost", port=5000)
  \end{minted}
  \caption{Simple server implemented in Python using Flask.}
  \label{lst:server}
\end{listing}

\subsection{Communication with HomePort}
In order to communicate with HomePort, 
we need to do
\begin{enumerate}
  \item Map actions as strings, to actions as numeric values
  \item Send HTTP requests (\texttt{GET} and \texttt{PUT})
  \item Transform XML data from HomePort to JSON
\end{enumerate}

We have borrowed a box from Aalborg University,
containing two ordinary light bulbs, 
a Phidget interface kit and a relay.
This Phidget box has been used to test the implementation of connecting with HomePort. 

\subsubsection{Mapping Actions}
In the current implementation of HomePort, 
there is only two actions: $\{0, 1\}$.
We implement two actions: $\{\var{turnOff}, \var{turnOn}\}$. 
Actions are implemented differently on the server, 
in order to give each action a meaningful value, 
\eg ``turnOn'' is more descriptive than a ``0''.
The two sets of actions are implemented as a list of tuples, 
where the first value is the HomePort value, 
and the second value is our string value. 

To map the actions, we either send a numeric value (map to string) or a string value (map to numeric), 
and look up either the numeric or string value in our list of tuples of actions.

If the list of tuples is
\begin{center}
  \texttt{ACTIONS = [('0', 'turnOff'), ('1', 'turnOn')]},
\end{center}
then if we send \texttt{0}, we get \texttt{'turnOff'}. 

\subsubsection{Sending HTTP Requests}
We use another Python module called httplib, 
to send the requests from the server to HomePort. 
The request we send are either a GET or a PUT. 
The GET is simply a \texttt{GET /devices HTTP/1.1}. 

To perform an action, we use the aforementioned mapper to get the numeric value of our action and send a PUT request with 
\begin{listing}
\begin{minted}[breaklines, autogobble]{python}
"<?xml version='1.0' encoding='UTF-8'?><value>" + actionID + "</value>"
\end{minted}
\caption{Example of PUT request data.}
\label{lst:putrequest}
\end{listing}
as data. 

\subsubsection{Transforming XML Data}
When we request devices from HomePort, 
we receive an XML response. 
An example response of HomePort could be:
\begin{listing}
\begin{minted}[breaklines, autogobble, escapeinside=||]{xml}
<?xml version="1.0"?>
<devicelist name="services.xml" id="12345">
  <device desc="phidget" id="173111" vendorid="0x01" productid="0x01" version="V1" location="ExampleRoom" type="phidget">
    ...
    <service desc="input" id="0" value_url="/phidget/173111/output/0" type="input" unit="0/1">
      <parameter id="0"/>
    </service>
    |\ldots|
  </device>
</devicelist>
\end{minted}

\caption{Example of HomePort output.}
\label{lst:homeportoutput}
\end{listing}

The values that we are interested in here are the attributes of the \texttt{service} element, 
mainly the attributes \texttt{id}, \texttt{value\_url} and \texttt{unit}. 
The \texttt{desc} attribute has the potential to contain the name of the smart device, 
but is not so in the version of HomePort that we are using. 
We will use the \texttt{id} to identify the HomePort, 
the \texttt{value\_url} to easily send actions to the device, 
and \texttt{unit} to figure out what actions are available. 

The \texttt{unit} attribute is however not compatible with our actions, 
so we transform the string \texttt{''0/1''} to a list \texttt{['0', '1']}, 
and then use our mapper to convert these to our actions. 

We also need to get the state of the device. 
If we know the state of the device, 
we are able to use a single gesture for on and off,
\ie the action of the gesture could simply switch the binary state of the device. 
In HomePort, the state is either 0 (off) or 1 (on). 
To get the state of the device, 
we can do a \texttt{GET} request to the URL of the device, \eg:
\begin{center}
  \texttt{GET /phidget/173111/output/0},
\end{center}
and then map the state of HomePort to our states.

The names and coordinates of the \texttt{Device} objects must be inserted by the user and are currently hardcoded, 
as we have not implemented adding new devices. 

The JSON result of the transformed XML, with a name and coordinate is:

\begin{listing}
\begin{minted}[breaklines, autogobble, escapeinside=||]{text}
{
  "devices": [
  {
    "actions": ["turnOff", "turnOn"], 
    "coords": [6.5, 3.35], 
    "id": "0", 
    "name": "Lamp at Couches", 
    "state": "off", 
    "url": "/phidget/173111/output/0"
  }, 
  {|\ldots|}
  ]
}
\end{minted}
\caption{JSON output of our server}
\label{lst:serveroutput}
\end{listing}
