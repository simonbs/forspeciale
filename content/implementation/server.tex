\section{Server}\label{sec:serverimplementation}
The server was described in \Cref{sec:serverdesign}. 
In this section, we briefly document the implementation of the server.

The server is implemented in Python,
using a microframework called \texttt{Flask},
which provides a built-in server and a RESTful request handler, among other things. 
As described in \Cref{sec:serverdesign}, 
the server only provides two REST methods for controlling a devices and getting the devices. 
Using \texttt{Flask}, the server has been programmed as simple as \Cref{lst:server}, 
where the code for communicating with HomePort has been omitted. 
%Thalley: Er det for dumt at skrive dette? Skal koden fjernes, eller skal man indsætte mere kode?

\begin{listing}
  \begin{minted}[linenos, breaklines, autogobble]{python}
    from flask import Flask
    app = Flask(__name__)
    
    @app.route('/', methods=['POST'])
    def do_action():
      action = request.form['action']
      id = request.form['id']
      # Code to send request to HomePort
      
    @app.route('/devices', methods=['GET'])
    def get_devices():
      # Code to return devices from HomePort
    
    if __name__ == '__main__':
    app.run(host="localhost", port=5000)
  \end{minted}
  \caption{Simple server implemented in Python using Flask}
  \label{lst:server}
\end{listing}

\subsection{Communication with HomePort}
In order to communicate with HomePort, 
we need to do
\begin{enumerate}
  \item Map actions as strings, to actions as numeric values
  \item Send HTTP requests (\texttt{GET} and \texttt{PUT})
  \item Transform XML data from HomePort to JSON
\end{enumerate}

\subsubsection{Mapping Actions}
In the current implementation of HomePort, 
there is only two actions: $\{0, 1\}$.
We implement two actions: $\{\var{turnoff}, \var{turnOn}\}$. 
The two sets of actions are implemented as a list of tuples, 
where the first value is the HomePort value, 
and the second value is our string value. 

To map the actions, we either send a numeric value (map to string) or a string value (map to numeric), 
and look up either the numeric or string value in our list of tuples of actions. 

If the list of tuples is
\begin{center}
  \texttt{ACTIONS = [('0', 'turnOff'), ('1', 'turnOn')]},
\end{center}
then if we send \texttt{0}, we get \texttt{'turnOff'}. 

\subsubsection{Sending HTTP Requests}
We use another python module called httplib to send the requests. 
The request we send are either a \texttt{GET} or a \texttt{PUT}. 
The \texttt{GET} is simply a \texttt{GET /devices HTTP/1.1}. 

To perform an action, we use the aforementioned mapper to get the numeric value of our action and send a \texttt{PUT} request with 
  \begin{minted}[breaklines, autogobble]{python}
  "<?xml version='1.0' encoding='UTF-8'?><value>" + mapper.map_action_to_id(action) + "</value>"
  \end{minted}
as data. 

\subsubsection{Transforming XML Data}
When we request devices from HomePort, 
we receive an XML response. 
An example response of HomePort could be:
\begin{minted}[linenos, breaklines, autogobble]{xml}
<?xml version="1.0"?>
<devicelist name="services.xml" id="12345">
  <device desc="phidget" id="173111" vendorid="0x01" productid="0x01" version="V1" location="ExampleRoom" type="phidget">
    ...
    <service desc="input" id="0" value_url="/phidget/173111/output/0" type="input" unit="0/1">
      <parameter id="0"/>
    </service>
    ...
  </device>
</devicelist>
\end{minted}

The values that we are interested in here is the attributes of the \texttt{service} element, 
mainly the attributes \texttt{id}, \texttt{value\_url} and \texttt{unit}. 
The \texttt{desc} attribute have the potential to contain the name of the smart device, 
but is not so in the version of HomePort that we are using. 
We will use the \texttt{id} to identify the HomePort, 
the \texttt{value\_url} to easily send actions to the device, 
and \texttt{unit} to figure out what actions are available. 

The \texttt{unit} attribute is however not compatible with our actions, 
so we transform the string \texttt{''0/1''} to a list \texttt{['0', '1']}, 
and then use our mapper to convert these to our actions. 

We also need to get the state of the device. 
In HomePort, the state is either 0 (off) or 1 (on). 
To get the state of the device, 
we can do a \texttt{GET} request to the URL of the device, \eg 
\begin{center}
  \texttt{GET /phidget/173111/output/0},
\end{center}
and then map the state of HomePort (0 or 1) to our states (off or on). 

The names and coordinates of the \texttt{Device} objects must be inserted by the user. 
The names and coordinates in our implementation is currently hardcoded, 
as we have not implemented adding new devices. 

