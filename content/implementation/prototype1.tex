\section{Prototype \#1}
\label{sec:implementation:prototypes:prototype1}
The purpose of the first prototype 
was to test how to detect which devices users point at, 
given the user's position and the direction of the user.

\subsection{Description}
The first prototype was an Android phone application, 
that showed the devices that the user pointed at.
The application used a $10 \times 10$ grid to simulate a square room, 
in which the user and the devices to be controlled were located.
A set of virtual devices was hardcoded into the application with fixed positions, 
and the grid would look like the one shown in \Cref{table/prototype-grid}.
The user is positioned in the center at $(5,5)$ by default. 
This position can be changed by using the \texttt{Up}, \texttt{Down}, \texttt{Left} and \texttt{Right} buttons.
When the user's direction changes, 
a list of devices being pointed at is found by calculating the angle between the user's position, 
and the position of each device.
The direction is found using the magnetometer, \ie utilizing magnetic fields,
such that \num{0} is north, \num{-90} is west, \num{90} is east, \etc.
There is a possibility that speakers, wires and other magnetic fields, 
may interfere with the correctness of the direction. 
We have, however, not found any significant interference during the development of theses prototypes. 

The angle between each smart device and the user, 
is calculated by \Cref{eq:angle} from \Cref{sec:analysis:orientation}:
\begin{equation}
\var{angle} = 180 / \pi * \arctan(\var{user.y} - \var{device.y} / \var{user.x} - \var{device.x})
\end{equation}
where \var{user.x} and \var{user.y} are the $x$ and $y$ coordinates of the user and likewise for the device.
A screenshot of this application is shown in \Cref{fig:prototype1-app-screenshots}.

\begin{table}[!htb] 
    \centering
    \tiny
    \begin{TAB}(e){|c:c:c:c:c:c:c:c:c:c|}{|c:c:c:c:c:c:c:c:c:c|}
     &  &  &  &  &  &  &  &  & Stereo \\
     &  &  &  &  &  &  &  &  &  \\
     &  &  &  &  &  & \begin{tabular}[c]{@{}l@{}}Coffee\\ Maker\end{tabular} &  &  &  \\ 
     &  &  &  &  &  &  &  &  &  \\ 
    \begin{tabular}[c]{@{}l@{}}Garage\\ Door\end{tabular} &  &  &  &  &  &  &  &  &  \\ 
     &  & Lamp 2 &  &  &  &  &  &  &  \\
     &  &  &  &  &  &  &  &  &  \\ 
     &  &  &  &  &  &  &  &  &  \\ 
     &  &  &  &  &  &  &  &  &  \\ 
    Lamp 1 &  &  &  &  & TV &  &  &  &  \\
    \end{TAB}    
    \caption{A $10 \times 10$ grid showing the position of devices. Lamp 1 is located at $(0,0)$ and Stereo is located at $(9,9)$.}
    \label{table/prototype-grid}
\end{table}

\subsection{Results}

From the prototype we found that given the user's position, 
the user's orientation, and the positions of smart devices, 
we can determine which smart devices a user points at. 

Based on this we can expand the prototype, 
to work with actual devices, 
and perform actions on the smart devices being pointed at.

\begin{figure}[!htb]%
    \centering
    \subfloat{
        \includegraphics[width=0.3\textwidth]{images/Prototype1_Android_1.png}
    }
    \subfloat{
        \includegraphics[width=0.3\textwidth]{images/Prototype1_Android_2.png}
    }
    \caption{Screenshots of the first prototype.}
    \label{fig:prototype1-app-screenshots}
\end{figure}

%%% Local Variables:
%%% mode: latex
%%% TeX-master: "../../master"
%%% End:
