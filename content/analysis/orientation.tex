\section{Orientation}\label{sec:analysis:orientation}
In this project, a solution for pointing at a controllable device, \eg a lamp, is proposed. 
In order to determine which device the user points at, 
and thereby intends to control, 
it is necessary to determine the locations of the devices in the system, 
relative to the user. 
We must thus know the location of the user, 
and his controllable devices along with the orientation of the user.

The orientation of the user can be retrieved from any device possessing a magnetometer. 
We found that 49 devices in Vandrico's database contain the magnetometer component. 
Given the orientation of the user, his location and the location of a controllable device, 
we can calculate if the user is looking at, 
or in the direction of, the controllable device.
We also need to define a visibility range, \ie a range that constitutes the area in which items are visible to the system. This range can be specified (in degrees) as $r = o \pm e$, where $r$ is the visibility range, $o$ is the orientation of the user and $e$ is the distance between $o$ and the edges of $r$.
\Cref{fig:visibilityangle} illustrates this. 
In this figure, the user's orientation is \var{o1}, 
and there are two devices with orientations \var{o2} and \var{o3}. 
An orientation in degrees is between \num{0} and \num{360}. 
To determine whether a device with orientation \var{o2} is within the visibility range, 
it is a simple matter of calculating whether \var{o2} is greather than $\var{o1}-\var{e}$ and smaller than $\var{o1}+\var{e}$, with some exceptions when the visibility range spans the gap between 359\degree and 0\degree.

\begin{figure}[!htb]
    \centering
    \def\svgwidth{0.6\textwidth}
    \import{drawings/}{drawings/visibilityangle.pdf_tex}
    \caption{Finding objects in the visibility range. Device 1 is within the range, but Device 2 is not.}
\label{fig:visibilityangle}
\end{figure}

%%% Local Variables:
%%% mode: latex
%%% TeX-master: "../../master"
%%% End:
