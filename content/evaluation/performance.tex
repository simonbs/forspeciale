\section{Performance of Gesture Recognition}
Gesture recognition is a core part of our solution and it is being performed often. 
Since the gesture recognition is being performed by device with somewhat low performance,
it is important that we have high performance of our gesture recognition. 
In this section we test the performance of recognizing gestures. 
We test the performance by executing a number of gestures and see how fast we can recognize these.  
We expect the number of unique gestures in the database to increase the recognition time. 
Furthermore, the number of times each gesture has been trained also increase the complexity of recognizing gestures. %Thalley: Remove this line if that is no longer the case

For this setup, we have [NUMBER OF GESTURES], each trained [NUMBER OF TRAININGS] times. 
We start by evaluating with \num{1} gesture, and then increase the number of unique gestures to estimate the computational complexity.

\begin{figure}[!htb]
    \centering
    \todo[author=Thalley]{Insert figure}
    \caption{Graph showing the time of recognizing gestures, with increasing number of unique gestures}
    \label{fig:performancegraph}
\end{figure}
\todo[author=Thalley]{Perform performance of gesture recognition tests}

The result from \Cref{fig:performancegraph} show that the complexity of gesture recognition is [RESULT]. 

To find the time it takes to recognize a single gesture, 
we perform the recognition of a random sample of known gestures \num{1000} times, 
and calculate the mean time spend on recognizing each gesture. 
Since the number of unique gestures has an impact on the complexity, 
we choose a fixed number of common gestures each trained [NUMBER OF TRAININGS] times. 
We expect at common number of unique gestures to be [NUMBER OF COMMON GESTURES], 
and for that number of gestures we find that the average time to recognize a single gesture is [RESULT]. 

\subsection{Performance of Gesture Recognition Conclusion}
From the results, we can conclude that... \todo[author=Thalley]{Write conclusion of performance test based on results}
