\section{Performance of Gesture Recognition}\label{sec:gestureperformance}
Gesture recognition is a core part of our solution and it is being performed often. 
Since the gesture recognition is being performed by a device with somewhat low performance,
it is important that we have high performance of our gesture recognition. 
In this section we test the performance of recognizing gestures. 
We test the performance by gradually populating the database with gesture traces and see how fast we can recognize a random gesture input.
We expect the number of gesture traces in the database to increase the recognition time.

For this setup, we add five gesture traces at a time. 
This is to simulate training a gesture five times, 
as recommended by \cite{threedollar}. 
We then generate a random gesture input, 
and run the recognition function on it ten times, 
logging the execution time.
We repeat this until there is a total of \num{100} gesture traces in the database, 
which corresponds to \num{20} unique gestures.
We performed this test six times on an iPhone 5, 
with a \SI{1.3}{\giga\hertz} dual-core ARM processor and \SI{1}{\giga\byte} RAM.

\begin{figure}[!htb]
    \centering
    \begin{tikzpicture}
  \begin{axis}[
%      ybar,
%      bar width=2pt,
      xlabel = Number of unique gestures,
      ylabel = Average time in ms,
      xtick=data,
      width=0.95\textwidth,
      height = 6cm,
      yticklabel style={align=right,inner sep=0pt,xshift=-0.3em},
      enlargelimits = false,
      ymax = 150,
      grid=major,
      try min ticks=10]]
    \addplot table[x=gestureNo, y=time] {data/three-dollar-test-results/results/10xrecognize/average.csv};   
  \end{axis}
\end{tikzpicture}
    \caption{Graph showing the time of recognizing gestures, with increasing number of gesture traces. Each unique gesture is training \num{5} times.}
    \label{fig:performancegraph}
\end{figure}
\todo[author=Kasper]{Lige nu bruger vi bare resultatet af én af mange tests. Vi kan overveje at slå dem sammen og tage gennemsnitstiden.}

\subsection{Performance of Gesture Recognition Conclusion}
The result from \Cref{fig:performancegraph} shows that the time spent recognizing a gesture, 
increases linearly with the amount of gesture traces in the database.
As the system only attempts to recognize a gesture once, 
the performance of the ``Three Dollar Gesture Recognizer'' should be adequate for use in our system. 
