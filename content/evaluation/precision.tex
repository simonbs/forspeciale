\section{Precision of Indoor Location}\label{sec:estimoteprecision}
Another core part of our system is indoor location. 
For the ``point-to-select'' part of our system to work as intended, we need high indoor precision. 
In \Cref{sec:indoor-positioning} we mentioned that Estimote claims the accuracy to be less than \num{5} meters.
In this section we test if that is actually the case, or even if we can achieve better results than that. 
We test this by comparing the position we get from the application to the actual position we have in the room. 

We setup a [SIZE OF ROOM] room, with [NUMBER OF BEACONS]. 
The room is illustrated by \Cref{fig:precisiontest}. 
In \Cref{fig:precisiontest} you can also see small spots. 
These spots is the known locations where we are going to perform the precision tests. 
We randomly walk between these spots [NUMBER OF TIMES] and find the mean error rate.
\begin{figure}[!htb]
    \centering
    \todo[author=Thalley]{Insert figure}
    \caption{Illustration of room used for indoor location precision test}
    \label{fig:precisiontest}
\end{figure}

The mean error rate that we found is [RESULT]. 

\subsection{Precision of Indoor Location Conclusion}
From the results, we can conclude that... \todo[author=Thalley]{Write conclusion of precision test based on results}

