\section{Server and HomePort}\label{sec:servereval}
We have tested the correctness of the server, 
but not the performance.
We have omitted testing the performance, 
as the number of requests send to the server i minimal. 

To test the correctness of the server, 
we simulate \num{5} devices,
and send both GET and POST requests to the server.
We send a total of \num{10000} requests (a mix of POST and GET) to the server, 
and validate the response received. 
We send both valid requests, 
\ie \texttt{GET /devices} and POST with valid ID and action,
but also invalid POST requests, \ie POST requests with invalid actions. 
Each of the \num{10000} requests returned the expected response code, 
\ie 200 for valid requests, and 400 for bad requests.
We can conclude that the server works as intended.

However, our third party smart hub, HomePort, showed problems. 
The implementation of HomePort that we use, 
is the so-called OldHomePort\footnote{\url{https://github.com/home-port/HomePort/tree/OldHomePort}}. 
We used this version as it was the only one with a Phidget adapter implemented. 
Through the development of the server, 
we found that this version of HomePort crashes up to a few times every minute when handling requests. 
Sometimes it will even disconnect the Phidget interface kit, 
and thus all the devices, at seemingly random points of time. 
We have not worked on determining the errors of OldHomePort, 
nor have we tried to debug it. 
This is primarily because it is a third-party component of our system, 
and has only been used to properly test the communication between the devices and the phone. 