\section{Profit Influence Experiments}\label{sec:profitexp}
In this section, experiments will be conducted to determine how different inputs to scheduling will influence profit of a BRP with electric vehicle consumers. Different cases and the independent parameters are tested in the experiments, to determine how they influence the profit. The goal of these experiments is to determine how BRPs best can improve their profits and which scheduling method will give the highest profit.

The different input parameters will vary in terms of:

\begin{itemize}
	\item \textbf{Distance:} equal to how far an EV can drive using the specified percentage of its battery's maximum capacity. Higher percentage means longer distance and thus higher \var{minEnergyNeeded}.
	\item \textbf{Deadline:} how much the percentage of an EVs deadline must be used to be \var{minCharged}. Higher percentage means shorter deadlines.
	\item \textbf{Weather forecast error:} how erroneous a weather forecast is. If it is \SI[parse-numbers = false]{x}{\percent}, the actual production is between \var{-x} and \var{x} percent off. The more erroneous the forecast is, the less accurate and predictable the scheduling will be. 
	\item \textbf{Price forecast error:} is much like the previous value, just with price forecasting. The more erroneous the forecast is, the more the scheduling will be off when choosing the cheapest time to import energy. 
\end{itemize} 

We set up three different cases with an initial value of each of these parameters. The values of each of these parameters will be defined using spiderweb diagrams, as shown by \Cref{fig:case0spiderweb}. The value at the center of the diagram is always \perc{0} and the maximum values are \perc{100}, with the distance between each tick being \perc{10}.


When we change the values of these variables, the total profit of the BRP will change as well. When conducting these experiments, we will be able to see which variables have the highest influence on profit. We will use three cases with standard values for each of the variables. In each case, we will change the values of variables, one by one, and illustrate the results using charts.

It is important to note that the way we introduce weather forecasting and price forecasting errors will have a different result than if they were true forecasting errors. We implement these forecasting errors by using actual data, i.e.\ not forecasted data, and add predefined coefficients. For example, if we have a day with a time slot size of 20 minutes, then a total of \num{72} time slots exist during a day. Each time slot will have its own amount of production (in \kwh{}) and its own energy price (in \si{\euro}. For each of these amounts or prices, we add a coefficient corresponding to the error range we want. If we want a \perc{10} forecasting error, we add a value between \perc{-10} and \perc{10} to the originally observed value. The list of the coefficients can be seen in \Cref{app:coefficients}; the length of this list is twice the size of the amounts of time slots, because we will in reality have 144 time slots, as we at the 72nd time slot still schedule 72 time slots in the future. After having done these tests we realized a mistake with the coefficients, they do not add up to 0, this skews the results a bit, however they are still valuable.

A last note about the setup is that we assume importing energy is more expensive than what we can get from charging EVs or from exporting energy, as also stated in the optimization problem in \Cref{sec:idea}. We thus add coefficients to each type of energy transaction and define the costs to be: 
\begin{itemize}
  \item Importing: \num{1.1}
  \item Charging: \num{1}
  \item Exporting: \num{0.9}
  \item Negative imbalance (import): \num{1.2}
  \item Positive imbalance (export): \num{0.8}
\end{itemize}
The profit is then calculated using \Cref{eq:profit} from \Cref{sec:marketmodel}.

When we charge an EV in a time slot, we charge the user the amount equal to the current time slot. When we import or export energy for the future (using our schedule) we multiply the price with the defined costs. Because we have forecasting errors, we will also have imbalances when we e.g.\ forecast more production than we actually have. When we have these imbalances, importing or exporting energy will have a greater effect on the price as BRPs charge more for these because they have to transfer energy in real time. For these cases, we will always assume that when we do not schedule, i.e.\ we use the no scheduling method, we always use the imbalance prices. This will always put the no scheduling method at a disadvantage and the total profit for the no scheduling method in the following charts will be significantly lower.

Each experiment is run with two wind turbines, each producing on average \SI{237}{\kilo\watt} and 700 different EVs, where 200 of those EVs are plugged in from the beginning and the remaining 500 are added over the course of a day. We could conduct the experiments with more EVs instead of changing the distance, but this will not change the difficulty of the scheduling and only scale the consumption. This would not give an interesting result as it would simply just require more energy per time slot.

To illustrate how the scheduling works and to show the effects of scheduling compared to no scheduling we run a case 0, seen by \Cref{fig:case0spiderweb}. When run with the input shown in \Cref{fig:case0spiderweb}, the resulting chart is shown in \Cref{fig:case0_chart}. We will convert these types of charts into charts clearly showing profit of the BRP. 

\begin{figure}[!htbp]
	\centering
	\input{drawings/spiderwebdiagram}
	\caption{Spiderweb diagram for case 0}\label{fig:case0spiderweb}
\end{figure}

\begin{figure}[!htb]
	\centering
	\includegraphics[width=\linewidth]{drawings/testresults/case0.tikz}
  \caption{Example of a scheduled day with two wind turbines and \num{700} EVs for \protect\formatdate{2}{5}{2013}}\label{fig:case0_chart}
\end{figure}
Looking at \Cref{fig:case0_chart} the benefit of scheduling is clear. When the amount of power the EVs can consume is higher than the power produced, no scheduling uses too much power. When the amount of power the EVs can consume is less than the power produced no scheduling uses less power than the scheduler. The fact that no scheduling uses less power than the schedulers use is due to some EVs having been charged to their minimum needed capacity.    
As mentioned, we will perform experiments using three cases. We change the standard input parameters in each of the cases as seen by \Cref{tab:caseinputs}. Each case will represent a certain scenario which should be represented by the input parameters. 
\begin{table}[!htbp]
	\centering
	\begin{tabular}{l S S S s}\toprule
    Criteria                & {Case 1} & {Case 2} & {Case 3} & {Units} \\ \midrule
		Distance percentage       & 20     & 40     & 20 & \percent \\
		Deadline flexibility 			& 70     & 50     & 70 & \percent \\
		Weather forecast error 	  & 10     & 10     & 20 & \percent \\
		Production forecast error & 10     & 10     & 20 & \percent \\ \bottomrule
		\end{tabular}
	\caption{Table showing input percentages of each input parameter for test cases 1--3}\label{tab:caseinputs}
\end{table}

\subsection{Case 1: Normalized values} 
We have modeled the first case after what we estimate is a somewhat regular situation, i.e.\ where: 
\begin{itemize}
	\item Distances matches numbers from~\cite[p.~52]{TRAFIKSTYRELSEN2010}
	\item Deadline lengths match the time between early afternoon and morning 
	\item Forecasting errors are equal to what we assume
\end{itemize} 

The input we have for case 1 can be seen in \Cref{fig:case1spiderweb}. 
\begin{figure}[!htb]
	\centering
	\input{drawings/spiderwebdiagram_case1}
	\caption{Spiderweb diagram for case 1}\label{fig:case1spiderweb}
\end{figure}

Using the input shown in \Cref{fig:case1spiderweb}, we change each of the variables, one at a time, to see how the total profit changes. \Cref{fig:test_case1} shows the resulting profit changes. Each chart consists of 20 data points, each of which are a total sum of a whole day using our system.

\begin{figure}[!htb]
	\centering
	\input{drawings/testresults/case1.tex}
	\caption{Profit influence testing on case 1}\label{fig:test_case1}
\end{figure}

\subsubsection{Case 1 Conclusion}
Looking at the charts in \Cref{fig:test_case1} we can first of all conclude that the greedy scheduler and the LP scheduler are almost identical in terms of generating profit for the BRP. Most of the time, the two scheduling methods generate the same profit and it is only at a few different inputs that they differ. The interesting part here is that neither of the scheduling methods are clearly superior. 

Let us first take a look at \Cref{fig:case1_dist} and \Cref{fig:case1_deadline}. These two variables (distance and deadline flexibility) clearly shows the most interesting results. We see a clear point at around \perc{6} where no scheduling suddenly starts falling. This is because we here reach a distance where we start importing energy to keep up with the consumption, which becomes larger as we increase the distance. The three methods all flatten out as the distance becomes larger, because we will have to import more energy to satisfy the energy needed which will reduce the profit. In the case of no schedule, it flattens out because it reaches a point where all EVs will charge until their deadlines regardless of the distance, and the profit will thus not change as the deadline in this case stays the same. From \Cref{fig:case1_dist} we can conclude that there is a clear difference between scheduling and no scheduling as the distance becomes larger. We can also conclude that when we raise the distance to a larger value, the profit decreases, because energy needs to be imported to support the higher demand. The profit flattens out when we reach a distance where consumption becomes larger than production and the increased amount on energy we have to import makes it expensive to charge EVs.

From \Cref{fig:case1_deadline} we see a similar result. As we get more flexibility, and can thus generate better schedules, scheduling increases the profit for the BRPs. The profit from scheduling could be expected to always increase as flexibility increases. However, we actually see a lower profit with a flexibility of \perc{100} than we do with a flexibility of \perc{85}. This comes from the way we adjust the deadlines of the EVs and is caused by rounding, discussed in \Cref{sec:lpsched}. We set the EVs deadlines to be the current time plus a number of minutes we compute as following:
\[
  \left\lceil \frac{\var{timeSlotSize} \cdot \var{minTimeSlots}} { \left( \frac{(100-\var{deadlineFlexibility})}{100}\right) } \right\rceil
\]

where \var{timeSlotSize} is the amount of minutes in each time slot, \var{minTimeSlots} is how many time slots an EV must charge in to be \var{minCharged} and \var{deadlineFlexibility} is the deadline flexibility (in percentage). In case that \var{deadlineFlexibility} is \num{100}, we let the EVs' deadlines be \num{24} hours from the current time. 

It is an interesting result that the two scheduling methods may generate a worse schedule as flexibility increases, but it is also something that would only happen during tests as the system does not use the above formula when the system is actually running. Furthermore, we see a decrease in profit from no scheduling. This comes from the fact that a longer deadline (i.e.\ more flexibility) when not scheduling, will charge the EVs more than required, even if that means importing extra energy. It flattens out because at around \perc{80} flexibility all EVs will be \var{maxCharged} and thus, we can  not consume any more energy than what we can at \perc{80}. We can conclude that an increase in flexibility may increase or decrease the profit.

The last two charts in \Cref{fig:case1_prod} and \Cref{fig:case1_price} are slightly different from the first two. Recall that the error percentages are implemented using coefficients. In \Cref{fig:case1_prod} we see a decrease in profits as the error margin grows. This can be expected since we will have more imbalances, which are expensive and decrease the profit. \Cref{fig:case1_price} shows no increase or decrease in profit. This result may seem odd, but it simply shows that the price for energy does not have an influence on scheduling in this case. Recall that this case is almost identical to \Cref{fig:case0spiderweb} and the resulting scheduling is therefore almost identical to \Cref{fig:case0_chart}. As should be noticeable, energy is not imported nor exported in this case. Because we do not import or export much, scheduling is not affected by price forecasting errors. From these charts, we can conclude that production forecasting errors decrease the profit, as we will have to import or export energy which minimizes the maximum profit we can get from charging EVs. We can also conclude that price forecasting errors do not influence the profit if if there is no underproduction, i.e.\ we do not need to import energy.

For case 1 we can conclude that production forecasting errors and driving distance required have the highest influence on the BRP's profit. To see the actual difference between the scheduling methods and no scheduling, we sum the profit of the 20 data points in each chart and then sum the profit each chart for each scheduling method. The result of this can be seen in \Cref{tab:case1_total}.
\begin{table}[!htb]
	\centering
	\begin{tabular}{l c c c}\toprule
		           & No scheduling & Greedy & LP \\ \midrule
    Total sum  & \EUR{66884.70} & \EUR{75235.42} & \EUR{75298.65} \\ 
		Difference & -- & \perc{12.49} & \perc{12.58} \\ \bottomrule
	\end{tabular}
	\caption{Calculated difference between the scheduling methods in case 1.}\label{tab:case1_total}
\end{table}

Considering the charts in \Cref{fig:test_case1}, it was expected that the greedy scheduling method and the LP method would give a higher profit than no scheduling, but also that they would be very close. We can conclude that using these scheduling methods will improve the profit of a BRP by \perc{12.49} to \perc{12.58} in a case such as case 1. 

\FloatBarrier
\subsection{Case 2: High demand, low flexibility}
In case 2 we increase the value for distance and decrease deadline flexibility. This case should resemble a scenario where people have to drive farther every day and plug in their vehicles late. The input we have for case 2 can be seen in \Cref{fig:case2spiderweb}. 

\begin{figure}[!htb]
	\centering
	\input{drawings/spiderwebdiagram_case2}
	\caption{Spiderweb diagram for case 2}\label{fig:case2spiderweb}
\end{figure} 

We use the same method as case 1 to experiment on the input, and the resulting charts can be seen in \Cref{fig:test_case2}.

\begin{figure}[!htb]
	\centering
	\input{drawings/testresults/case2.tex}
	\caption{Profit influence testing on case 2}\label{fig:test_case2}
\end{figure}

\subsubsection{Case 2 Conclusion}
The resulting charts in \Cref{fig:test_case2} do not differ much from the charts in \Cref{fig:test_case1}, as could be expected, since we do not change the input significantly. The main difference between the two first cases is that in case 2, we see a small change in profit of \Cref{fig:case2_price} when altering the price forecasting error. From \Cref{fig:case2spiderweb} one can see that we have increased the distance percentage and decreased the deadline flexibility, making it both harder to schedule but also increasing the energy consumption. This will require that we import energy from the external grid and the price forecasting errors will thus have an influence on the total profit. We can conclude that price forecasting errors have a higher influence the more energy we have to import. For case 2 we can thus conclude that price forecasting errors have a higher influence when we have a higher energy demand where we have to import, which is an expected result. The actual differences between the schedulers and no scheduling in case 2 are shown in \Cref{tab:case2_total}. In this case, we see a higher profit increase than in case 1 due to the increased energy demand. We can thus conclude that a higher demand results in a larger difference between not doing any scheduling and using a scheduler.  

\begin{table}[!htb]
	\centering
	\begin{tabular}{l c c c}\toprule
           & No scheduling & Greedy & LP \\ \midrule
Total sum  & \EUR{66209.94} & \EUR{75356.17} & \EUR{75385.33} \\
Difference & -- & \perc{13.81} & \perc{13.86} \\ \bottomrule
	\end{tabular}
	\caption{Calculated difference between the scheduling methods in case 2.}\label{tab:case2_total}
\end{table}

\FloatBarrier
\subsection{Case 3: Low forecasting accuracy} 
We set the forecasting errors to be higher in this third and final case. This scenario could arise in a number of ways, e.g.\ weather can be harder to forecast based on the geographical positions' climate. The forecasting of prices, are influential by many factors and can be very unstable based e.g.\ power plant outages or based on lower energy than expected from RES. The inputs we have for case 3 can be seen in \Cref{fig:case3spiderweb}. The resulting charts are very much like the charts in \Cref{fig:test_case1} and are thus only in \Cref{app:case3}.

\begin{figure}[!htb]
	\centering
	\input{drawings/spiderwebdiagram_case3}
	\caption{Spiderweb diagram for case 3}\label{fig:case3spiderweb}
\end{figure}

\subsubsection{Case 3 Conclusion}
Even thought the resulting charts are much like case 1, we still see a difference in the final profit results as can be seen by \Cref{tab:case3_total}. If we compare those results to the results in case 1 (\Cref{tab:case1_total}) and case 2 (\Cref{tab:case2_total}), we see a noticeable decrease in profit. This can be expected as we have increased the error margins and should thus get a lower profit due to extra import and export. An interesting thing about this result is that in this case the greedy scheduler performs slightly better than the LP scheduler. The difference is, however, very small and we cannot conclude anything from it without more test results, except that forecasting errors clearly decrease profit. 

\begin{table}[!htb]
	\centering
	\begin{tabular}{l c c c}\toprule
           & No scheduling  & Greedy        & LP \\ \midrule
Total sum  & \EUR{66087.18} & \EUR{73963.94} & \EUR{73914.27} \\
Difference & --             & \perc{11.92}   & \perc{11.84} \\ \bottomrule
	\end{tabular}
	\caption{Calculated difference between the scheduling methods in case 3}\label{tab:case3_total}
\end{table}

 \FloatBarrier
\subsection{Profit Influence Testing Conclusion}
For the sake of overview, \Cref{tab:caseComparison} below aggregates the main results of each scheduling method, highlighting the fact that the two methods are not very different. Both schedulers solve the exact same optimization problem under a tenth of percent differently.

\begin{table}[htbp]
  \centering
  \begin{tabular}{l S S S s}\toprule
    Scheduler          & {Case 1} & {Case 2} & {Case 3} & Units \\ \midrule
    Greedy             & 12.49 & 13.81 & 11.92 & \percent \\
    LP                 & 12.58 & 13.86 & 11.84 & \percent \\ \midrule
    Difference         & 0.09  & 0.05  & 0.08  & \percent \\ \bottomrule
  \end{tabular}
  \caption{Overview of the percentage increase in profit when compared to not scheduling, for both types of scheduling methods}
  \label{tab:caseComparison}
\end{table}

From our tests we can clearly see an increase in profit at around 12-\perc{13} using the three specified cases. The tests also indicate that there is not a clear winner between the greedy scheduler and the LP scheduler and both is viable in terms of increasing profits for BRPs.

In the following section we will test the schedulers with regard to their performance with large scale inputs.

