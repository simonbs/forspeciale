\chapter{Implementation}\label{chap:implementation}
This chapter focuses on how the actual system is implemented, highlighting some of the most vital parts of the system. The motivation for implementing our solution to the problem, is to evaluate on the feasibility of the components designed in \Cref{chap:design}. Pseudocode for implemented algorithms described in the analysis will be shown for some components. Listings in this chapter are directly from the source code and therefore written in Java, however some of the listings are modified to leave out irrelevant implementation. A basic understanding of any object-oriented programming language is sufficient to understand them. 

Because our solution deals with problems that are time based, such as scheduling every 20 minutes, we implement a simulator which can simulate a real world environment a lot faster than if it was real-time. The implementation for the simulator is described in \Cref{sec:simulator}. \Cref{sec:prosumer} briefly explains the implementation of our production (wind turbines), and consumption (EVs). We have implemented a way to forecast prices for energy, which will be described in \Cref{sec:priceforecasting}. Then in \Cref{sec:schedimpl}, we describe the implementation of different schedulers before evaluating the entire system in \Cref{chap:evaluation}.

\lipsum