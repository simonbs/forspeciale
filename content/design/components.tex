\section{Component Design}\label{sec:components}
This section will design the components and data structures needed in our system. These components will be visualized using UML diagrams and we will implement these components based on the diagrams. 

\subsection{Producers}
As mentioned previously, we will only focus on having wind turbines as electricity producers. We design our wind turbines to be as simple as possible and will only include what is minimally needed to compute how much energy that can be produced. \Cref{sub:Forecasting: Energy supply} will describe how we compute the energy produced from wind turbines. The wind turbine class can be seen in \Cref{fig:UMLWT}:

\begin{figure}[!htb]
	\centering
	\includegraphics{drawings/windturbine.tikz}
	\caption{The WindTurbine class}\label{fig:UMLWT}
\end{figure}

In \Cref{fig:UMLWT}, \umlvar{location} is the physical location of the wind turbine. \umlvar{bladeLength} is the length of the rotor blade of the wind turbines. \umlvar{efficiency} is how efficient the wind turbine is at converting wind energy into electric energy, measured in percentage. \method{getWindSpeed}{now} gets the wind speed for a specific point of time. \method{getProduction}{} reads the wind turbine for the production at that given time, in \si{\watt}.

\subsection{Consumers}
The consumers we have in our system are EVs. \Cref{fig:UMLEV} below shows the EV class and its corresponding user profile in \Cref{fig:UMLprofile}. This profile encapsulates the consumer's settings for their driving habits.

\begin{figure}[!htb]
	\centering
	\begin{minipage}[t]{0.49\textwidth}
		\centering
		\includegraphics{drawings/EV.tikz}
		\caption{The ElectricVehicle class}\label{fig:UMLEV}
	\end{minipage}
	\begin{minipage}[t]{0.49\textwidth}
		\centering
		\includegraphics{drawings/profile.tikz}
		\caption{The UserProfile class}\label{fig:UMLprofile}
	\end{minipage}
\end{figure}

In \Cref{fig:UMLEV}, \umlvar{id} is a unique ID for the EV so that the system can recognize the individual EV. \umlvar{chargingSpeed} is how fast an EV can charge. \umlvar{batteryCapacity} is the energy capacity of the battery, where \umlvar{currentEnergy} is how much energy is currently in the battery. \umlvar{kmPrkWh} is how many kilometers the EV can drive per \kwh{}. \method{charge}{minutes} charges the EV for a given amount of minutes. \umlvar{profiles} is a list of user profiles that are used to say how far the EV should drive on a given day, and can be added to the EV's list using the \method{addUserProfile}{distance, deadline} method. \Cref{fig:UMLprofile} shows the profile class that are used in the EV class. A profile is a data structure to hold the data of a specific day, the data being how far the EV drives that day and when it must be charged by. 

\subsection{Charge-offers}\label{sec:charge-offers}

Charge-offers are required because they are objects that encapsulate information about the consumer, their vehicle, and their future driving plans. Charge-offers are much like the flex-offers from \textsc{mirabel} and are the main component that allows us to have flexible consumption. \Cref{fig:UMLCO} shows how we have designed the charge-offers.

\begin{figure}[hptb]
	\centering
	\includegraphics{drawings/chargeoffer}
	\caption{The ChargeOffer class}\label{fig:UMLCO}
\end{figure}

Information about the deadline is required to make scheduling possible, so the scheduler knows how to distribute charging amongst the collected charge-offers. Each charge-offer is connected to a specific vehicle by the EV's ID. The \umlvar{deadline} is given by the charge-offer and is chosen by the owner of the EV when setting a profile. \umlvar{chargingSpeed} can be read from the EV whenever it is plugged in and is limited to either the charging station's or the EV's maximum charging speeds. The \umlvar{minEnergy} amount can be calculated by knowing how far the EV must drive, how much energy is in the battery already, and how many kilometers the EV can drive per \kwh{}. The number of kilometers we have to drive, the energy in the battery and how many kilometers per \kwh{} the EV can drive is also known variables, and the minimum energy amount is thus also known. \umlvar{maxEnergy} is the maximum energy that can be charged to the battery, and can be calculated as the difference between current battery energy and the battery capacity. In case that \umlvar{minEnergy} is greater than \umlvar{maxEnergy}, it is impossible to charge the EV to satisfy the distance needed, and the system should give an error. Formally the minimum energy required and the maximum energy we can charge are defined by the following equation:

\begin{subequations}
  \label{eq:minmaxEnergy}
  \begin{align}
    \var{minEnergy} & = \frac{\var{distance}}{\var{kmPrkWh}} - \var{currentEnergy} \label{eq:minEnergy} \\
    \var{maxEnergy} & = \var{batteryCapacity} - \var{currentEnergy} \label{eq:maxEnergy}
  \end{align}
\end{subequations}

All of this information is extracted from a reference to the charge-offer's EV. A reference to the corresponding EV is need both for retrieving more information about the EV, and sending charge-assignments. A charge-assignment is simply a message telling the EV to charge for some time, thus not requiring a data structure. 

The distance that can be driven per \kwh{} varies on lots of factors such as: driving type (highway or city), speed, status of the air conditioning system, drag from windows being down, and so on. We choose not to take these variables into account, even though they may have an impact. These factors would be trivial to take into account in a more advanced system by asking the driver for more information, or by monitoring the driver's behavior on the road, measuring the current temperature to determine the climate control settings, etc. We will \emph{not} evaluate these variables in our system.

In the following section we will review the information represented by the BRP.

\subsection{Balance Responsible Party Agent}\label{sec:brp}
Our system is designed as a multi agent system, as can be seen by \Cref{fig:scenario}. The agents in the system are the BRP and all the intelligent charging stations. Since both the BRP and the stations should be able to run and act on their own, they can each defined as an \emph{autonomous agent}. According to \cite{franklin1997agent}, which compares and synthesizes different definitions of an agent from IBM, MIT, and Apple, an autonomous agent differs from programs in general in that they ``sense their environment and act autonomously upon it''. We define an autonomous agent by \Cref{def:autoagent}.

\begin{definition}[Autonomous agent]\label{def:autoagent}
  An autonomous agent is a system situated within and a part of an environment that senses that environment and acts on it, over time, in pursuit of its own agenda and so as to effect what it senses in the future.\hfill \text{\cite{franklin1997agent}}
\end{definition}

Since our system's goal is to help the BRP, the system will focus exclusively on the BRP agent.

In our case, to simulate a real world environment, we will have to simulate the behavior of a BRP. The main interest of the BRP is to profit from balancing energy in its respective balancing area. To facilitate this objective, the BRP agent's agenda is to schedule and charge the consumers' EVs in the most efficient manner. It achieves this based on various sources, such as the electricity market, its EVs, and production of its wind turbines. This is done in pursuit of optimizing the profit for the BRP along with minimizing imbalances for the BRP. We have designed the BRP agent to facilitate the following data structures and operations: 

\begin{figure}[!htb]
  \centering
  \includegraphics{drawings/BRP.tikz}
  \caption{The BRP class}
  \label{fig:UMLBRP}
  	\begin{minipage}[t]{0.49\textwidth}
  		\centering
  		\includegraphics{drawings/chargecontext.tikz}
  		\caption{The ChargeContext class}\label{fig:UMLCC}
  	\end{minipage}
  	\begin{minipage}[t]{0.49\textwidth}
  		\centering
  		\includegraphics{drawings/chargeplan.tikz}
  		\caption{The ChargePlan class}\label{fig:UMLCP}
  	\end{minipage}
\end{figure}

The BRP is responsible for scheduling and charging EVs through their corresponding charge-offers. \umlvar{currentPlan} is a list of the charge-offers that have yet to be charged and should be charged next time the system sends out charge-assignments. \umlvar{energyExchanges} is a list of how much energy should be imported or exported in each time slot. \umlvar{profit} is the BRP's current profit, positive or negative. \umlvar{context} is a charge-context, shown in \Cref{fig:UMLCC}. This context is used for the \method{schedule}{context} method, which produces a charge-plan, shown \Cref{fig:UMLCP}. The \method{chargeEVs}{when} method charges all the charge-offers that are in the \umlvar{toCharge} list. \method{timeSlotProfit}{currentPrice} computes the profit for a time slot and adds the resulting profit to the \umlvar{profit} variable. \method{collectFuturePrices}{amount, when} forecasts future prices and returns a list containing those prices.

A charge-plan is the outcome of the \method{schedule}{context} method and contains two lists: \umlvar{energyExchange} and \umlvar{chargePlan}. \umlvar{energyExchange} is a list of how much should be imported or exported in the different time slots and is used in the BRP. \umlvar{chargePlan} is the charge-plan itself and is a list of lists, where the sublists represent time slots and contain charge-offers that should be charged in those time slots. This is the essence of the entire system.

The charge-context is the context in which the scheduling takes place. It contains the \umlvar{production} list, which is a list of how much power is forecasted to be produced in each time slot. Likewise, \umlvar{prices} contains the forecasted electricity prices for each time slot. \umlvar{chargeOffers} is a list of the current charge-offers of vehicles that are currently plugged in. 

With the agent represented as a class, we can begin to design methods to gather the required information and fulfill the operations of the BRP, resulting in a system that can be tested and monitored. The following section will describe how we select methods of forecasting electricity prices and production.
