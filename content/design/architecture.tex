\section{Architecture}
The system consists of three layers as illustrated in \Cref{fig:architecture}. 
Each layer can communicate with the layer below and above it if they exist.

The first layer consist of the wearable device and the mobile device. 
The two may be tightly coupled as some wearable watches, 
for example the Apple Watch, relies heavily on the phone. 
The two devices are responsible for listening to new movement data from the wearable, 
\eg new data from the accelerometer and the magnetometer.
The movement data is used for recognizing gestures, 
and the magnetometer data is used for determining direction, 
such that we are able to determine what devices are being pointed at. 
The topmost layer is also responsible for communicating with the beacons, 
in order to retrieve the position of the user, 
and possibly the position of the devices the user can control. 

The topmost layer communicates with a central server developed specifically for this project. 
The server is responsible for storing positions of the smart devices, 
as well as receiving actions, based on interpreted gestures, from the wearable device.
Whenever the wearable device recognizes a gesture, 
the wearable devices sends an action and a device ID to the server.
The server then relays this request to a smart hub. 

The smart hub is responsible for managing the devices that can be controlled using gestures. 
The smart execute the request received from the server, 
on the appropriate smart device. 

\begin{figure}[h]
  \centering
  \begin{tikzpicture}
    \node[anchor=center] at (-0.8,0.5) {(1)};
    \node[anchor=center] at (-0.8,-1.5) {(2)};
    \node[anchor=center] at (-0.8,-4) {(3)};
    
    \node[anchor=center] at (2.5,0.5) {Wearable Device};
    \node[anchor=center] at (7.5,0.5) {Mobile Device};
    \draw[thick] (0,0) rectangle (10,1);
    \draw[thick, dashed] (5,0) -- (5,1);
    \draw[thick,->] (10.85,0.5) -- (10.15,0.5);
    \draw[thick] (11,0) rectangle (13,1) node[pos=.5] {Beacons};
    
    \draw[thick] (0,-1) rectangle (10,-2) node[pos=.5] {Server};
    \draw[thick,<->] (5,-0.15) -- (5,-0.85);
    
    \node[anchor=center] at (1.2,-3.4) {SmartThings};
    \draw[thick] (0,-3) rectangle (10,-5);
    \draw[thick,<->] (5,-2.15) -- (5,-2.85);
    \draw[thick] (0.2,-3.8) rectangle (2.7,-4.8) node[pos=.5] (bulb) {Bulb};
    \draw[thick] (2.9,-3.8) rectangle (5.4,-4.8) node[pos=.5] (lock) {Lock};
    \draw[thick] (7.3,-3.8) rectangle (9.8,-4.8) node[pos=.5] (door) {Thermostat};
    \node at ($(lock)!.5!(door)$) {\ldots};
\end{tikzpicture}
  \caption{Architecture of the system.}
  \label{fig:architecture}
\end{figure}

When utilizing a smart hub, users must install a hub in their home. 
The hub communicates with the users devices, \eg bulbs, locks and thermostats. 
For privacy reasons, it may be interesting to move the logic of the server to the smart hubs in the users home. 
This would also remove a layer in the architecture, simplifying it.

%Thalley: Let's keep it general and not talk about the SmartThings Hub here. 
%However, Samsung does not allow developers to run applications on the Hub, 
%nor do they provide a database or any other means of storing larger amount of data in an application running in the SmartThings environment.
%An alternative is to develop our own hub and system for automating and communicating with the devices, 
%however, that would a yet another solution for solving that problem. 
%Many big companies provide such solutions including Google, Microsoft, Apple and Samsung.
%
%On the other hand, storing the logic on the server allows, 
%rather than a hub eliminates the need for distributing updates to multiple hubs every time a new feature is introduced or a minor bug is fixed. 
%Such changes only need to be deployed to a single server.

%%% Local Variables:
%%% mode: latex
%%% TeX-master: "../../master"
%%% End:
