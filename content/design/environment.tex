\section{Environment \& Target Group}
The solution of controlling smart devices with a wearable, 
could in practice be applied in a couple of given environment.
It could be used to control smart devices in \eg hospitals, warehouses, malls or even factories. 
However we feel that focusing on smart homes, 
will result in a system that is more relatable to people, and to us.
This also means that we have to work in a smaller setting,
than the aforementioned alternatives. 
Because we have a smaller area to cover (house compared to \eg a factory),
we also have to focus more on accuracy. 
Furthermore, design plays a more important role,
as potential consumers typically have higher requirements for design and interface,
than users of business solutions. 
This also means that the components we use as part of our system,
should also be components that we expect, or can expect, 
regular consumers to use on a regular basis. 

\subsection{Target Group}
Our proposed system can have various users. 
In this report so far, we have focused on smart homes, 
which correspond to a target group of people with smart homes. 
We argue that this group will be our main target group, 
as it is most likely to be used by people in that group for convenience of controlling their smart homes. 
People in this group will currently only be early adopters of technology, 
as the technology (smart homes/devices and wearables) has not yet become common in households.
Based on the trend of IoT, we think that this will no longer be the case in the near future (5-10 years). 

%Thalley: Is this too much repitition or irrelevant? 
However, we also envision other groups to be our potential target groups.
One target group could be people whose work requires using an electronic, 
while not being near it or not wanting to touch it due to the situation,
\eg doctors operating and wanting to use e.g. a device to get further information.

It will also be able to help handicapped people. 
If someone is unable to walk, 
giving that person an easy way to control devices (that may even be out of reach) from afar, 
could provide a better life situation. 
In the same context, by using visual and/or vibrational feedback, 
the system could give blind people a way to determine what he or she is pointing at. 

The concept could even work in larger areas such as cities. 
Imagine being able to point at some building, or in some direction, 
and then receive a message what building that is or what is in that direction. 

The system would of course need to be modified or altered, 
so that these target groups would be able to use it, 
but we think that the concept is applicable in these situations. 