\section{Environment \& Target Group}

The solution of controlling smart devices with a wearable device could be applicable in a variety of environments including but not limited to: Hospitals, warehouses and homes.
Each of these environments poses a different set of challenges and requirements to the solution, for example a home environment would be a more confined space compared to a warehouse and as such the precision of the solution becomes a more important factor.

We decided to focus on smart homes as we feel that it would result in a system that is more relatable to people, and to us.

\subsection{Target Group}
Our proposed system can have various users but since we have focused on smart home environments, 
our target group would be users thereof. 
We argue that this group will be our main target group, 
as it is most likely to be used by these people for convenience of controlling their smart homes. 
People in this group will currently only be early adopters of technology, 
as the technology (smart homes/devices and wearables) has not yet become common in households.
Based on the trend of IoT, we think that this will no longer be the case in the near future (5-10 years). 

It will also be able to help handicapped people. 
If someone is unable to walk, 
giving that person an easy way to control devices (that may even be out of reach) from afar, 
could provide a better life situation. 
In the same context, by using audio and/or vibrational feedback, 
the system could give blind people a way to determine what he or she is pointing at. 
