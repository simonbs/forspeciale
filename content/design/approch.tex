\section{Approach}\label{sec:approach}
In our problem statement in \Cref{sec:researchstatement} we asked what we can do with wearables in a smart home setting.
In the sections before that, we described some of the ways of controlling a smart home. 
One of these was using gestures to control devices in the home. 
As the most used sensor in wearables is the accelerometer (see \Cref{fig:wearables-sensors}), 
it makes sense to utilize this sensor to perform the gestures used to control the devices. 
Our approach to our problem statement is thus using a wearable, 
and its sensors, to perform gestures and thereby control a smart home. 

The goal of our project is thus to create a system that allows gesture-based communication between the user and smart devices.
We want to design the system so that a user, wearing a \emph{wrist wearable}, 
can simply point at a smart device and perform a gesture to control this device. 
This is much like the solution that Reemo (see \Cref{sec:smarthomecontrol}) uses. 
However, we feel that the line-of-sight requirement is a big loss of control as it severely limits the control of objects.
With a line-of-sight requirement, it is impossible to control objects in other rooms, 
such as turning on your coffee machine while you are in your bedroom. 
We want to remove this limitation in our solution. 

For this to work, we need a way to determine when a gesture is being performed and which device should perform the action. 
Gesture recognition is a large subject and will be described in \Cref{sec:gesturerecognition}. 
Furthermore, to determine which device should perform the action, 
without a line of sight requirement, we need to know the \emph{indoor location} of the user, 
the position of the device(s) and the heading of the gesture. 
We have already discussed indoor location in \Cref{sec:indoor-positioning}, 
but will elaborate on how to use it for our project in \Cref{sec:designindoorlocation}.
At last, as described in \Cref{sec:smarthomes}, 
we need a smart hub to allow communication between the different devices, 
including the wearable, to be able to perform actions on the devices. 
This approach thus requires accurate indoor location, a wrist-worn wearable that can recognize gestures and a hub for interoperability between the wearable and the smart devices. 

\subsection{Indoor Location}\label{sec:designindoorlocation}
In order to determine which device the user points at and thereby intends to control, 
it is necessary to determine the locations of the devices in the system relative to the user.

The focus of this project is not to position devices and as such it was not the intention to spend time developing an entire solution for positioning devices indoors. 
Instead it was desired to find an existing product that could be used to facilitate indoor positioning.
The solution should be available in the early phases of the project in order to start building the system based on the solution for positioning.

Ideally users of this project should be able to control any device that fits within the concept of Internet of Things he owns, 
the price for any device needed to position each controllable device should be low. 
If a user owns several devices that can be controlled using gestures and an extra device is needed for each in order to preform the positioning, 
the price of such a device should be at a minimum.

It is assumed that users already own one or more devices that fit within the concept of Internet of Things and possibly are early adapters of such technology, 
it is assumed they have some technological expertise. 
However, it easy to imagine that this project can be used in an office environment where employees of varying technological expertise work or in health care. 
Therefore users may have a varying degree of technological expertise and it should be easy to extend the solution with new controllable devices.

Naturally the accuracy of the solution used for indoor positioning plays an important part. 
\Cref{fig:indoor-positioning:incorrect} shows the consequence of an incorrect location. 
If a lamp is estimated to be at another location that it is actually located, 
the user must point to an incorrect location in order to control the lamp.
Furthermore if the estimate is too wide, that is, the given area in which the lamp is located is very big, 
there is a greater risk that locations overlap. 
Overlapping locations causes a complexity as it is necessary to determine which device the user desires to control if he points at the overlap as visualized in \Cref{fig:indoor-positioning:overlap}.

\begin{figure}[!htb]
    \centering
    \begin{minipage}[t]{0.45\textwidth}
        \centering
        \includegraphics[width=0.6\textwidth]{images/incorrect-positioning-estimate.png}
        \caption{Incorrect location estimate. The estimate is visualized as a striped circle.}
        \label{fig:indoor-positioning:incorrect}
    \end{minipage}\qquad
    \begin{minipage}[t]{0.45\textwidth}
        \centering
        \includegraphics[width=0.6\textwidth]{images/positioning-overlap.png}
        \caption{Overlap of estimated positions. The estimates are visualized as a striped circle.}
        \label{fig:indoor-positioning:overlap}
    \end{minipage}
\end{figure}

Based on the above the following criteria for assessing potential solutions can be outlined.

\begin{itemize}
    \item Availability
    \item Price
    \item Ease of use
    \item Accuracy
\end{itemize}