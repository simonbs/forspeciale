\section{Solution Approach}\label{sec:solapproach}
This section will introduce the solution we want to develop in order to answer the problem statement, defined in \Cref{sec:problem_statement}. The problem with RES is that a lot of energy is wasted because produced energy is not always entirely consumed. This problem is illustrated in \Cref{fig:unschedeng}. 

\begin{figure}[!htb]
	\centering
	\includegraphics[width=\textwidth]{drawings/nosch.tikz}
	\caption{Mismatch between production and consumption}\label{fig:unschedeng}
	\includegraphics[width=\textwidth]{drawings/sched.tikz}
	\caption{Match between production and consumption}\label{fig:schedeng}
\end{figure}

In \Cref{sec:smartgrids}, we mentioned that smart grids add a communication layer to an already existing electrical energy grid. These smart grids make it possible to send messages through the grid and allows control of electricity consumption remotely. We will develop a system that takes advantage of this new technology. An already existing system, \textsc{mirabel}, uses an approach where they take advantage of flexible energy and schedules it using flex-offers, as described in \Cref{sec:MIRABEL}. 

We will adopt a similar approach to scheduling flexible energy consumption, as the one used by \textsc{mirabel}. Our system should use a data structure similar to flex-offers to schedule flexible energy. As mentioned, we want to focus on working with wind turbines and EVs to keep everything RES-based. In order to schedule the flexible consumption by EVs, we need to know how much energy is produced in the future and then schedule according to that. Our system should thus use the technology provided by smart grids to control the consumption of charging EVs to make sure that it minimizes imbalances and maximizes profit for a BRP, the target group of such a system. The solution to the problem in \Cref{fig:unschedeng} should be to schedule the energy consumption, so that we get a match between production and consumption as shown by \Cref{fig:schedeng}. The main difference between our approach and \textsc{mirabel} is that we wish to reschedule the flexible data structure to account for changes in the schedule, for example due to more EVs being plugged in over the day. 

The following section will introduce the environment, in which this system should exist. We will describe how we expect the system to be used and in which context it exists.
